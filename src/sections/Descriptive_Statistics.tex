\section{Descriptive Statistics}
\subsection{Arithmetic Mean And Empirical Variance}

\ptitle{Arithmetic mean}
\begin{equation*}
    \bar{x}=\frac{1}{n}\sum_{i=1}^{n}x_i
\end{equation*}

\newpar{}

\ptitle{Empirical variance}
\begin{equation*}
    s^2=\frac{1}{n-1}\sum_{i=1}^{n}(x_i-\bar{x})^2
\end{equation*}

\newpar{}

\ptitle{Empirical standard deviation}
\begin{equation*}
    s=\sqrt{s^2}
\end{equation*}

\subsection{Ordered Samples}

Data sorted in increasing order. The position of an observation in the ordered data is called \textbf{rank}. If some observations have the same value, they are assigned their average rank.

\subsection{Empirical Quantiles}
$q_\alpha$ is the value so that approximately $\alpha$ (in $\%$) of the data points are smaller than $q_\alpha$.
\newpar{}
if $\alpha n \notin \mathbb{N}$, then
\begin{equation*}
    q_\alpha = x_{(\lceil\alpha n\rceil)}
\end{equation*}

if $\alpha n \in \mathbb{N}$, then
\begin{equation*}
    q_\alpha = \frac{x_{(\alpha n)}+x_{(\alpha n+1)}}{2}
\end{equation*}

\newpar{}
\ptitle{Special Quantiles}
\begin{itemize}
    \item \textbf{median} $= 50\%$-quantile $q_{0.5}$
    \item \textbf{lower quartile} $= 25\%$-quantile $q_{0.25}$
    \item \textbf{upper quartile} $= 75\%$-quantile $q_{0.75}$
    \item \textbf{Interquartile range (IQR)} $= q_{0.75} - q_{0.25}$
\end{itemize}

\subsection{Histogram}
A Histogram displays the number of observations in each interval of length $h$ (subdivision of range).
The height of each bar is
\noindent\begin{equation*}
    \frac{|\omega|}{nh}
\end{equation*}
where $|\omega|$ is the number of observations in the interval.
\begin{itemize}
    \item The area over an interval corresponds to the relative frequency (similar to density).
    \item Histograms are normalizted.
\end{itemize}

\subsection{Box Plot}
\begin{center}
    \includegraphics[width = .7\linewidth]{boxplot.png}
\end{center}
\begin{itemize}
    \item \textbf{largest normal observation}: $x_i\leq q_{0.75}+1.5\cdot$IQR
    \item \textbf{smallest normal observation}: $x_i\geq q_{0.25}-1.5\cdot$IQR
    \item \textbf{outliers}: oberservations outside of\newline $[q_{0.25}-1.5\cdot\text{IQR},q_{0.75}+1.5\cdot\text{IQR}]$
    \item boxplots are rougher summaries than histogrms that can be used to compare different datasets for
          \begin{itemize}
              \item center of the data
              \item variability
              \item skewness
          \end{itemize}
\end{itemize}

\subsection{Empirical CDF}
\noindent\begin{equation*}
    F_n(x)= \text{number of } i<n \text{ such that } x_i<x
\end{equation*}

\subsection{Pairwise Observed Data}
For paired data, two-dimensional scatter plots can be used to visualize their distribution.

\subsubsection{Empirical Covariance and Correlation}
\ptitle{Empirical Covariance}

\noindent\begin{equation*}
    s_{xy}=\frac{1}{n-1}\sum_{i=1}^n(x_i-\bar{x})(y_i-\bar{y})
\end{equation*}

\ptitle{Empirical Correlation}

\noindent\begin{equation*}
    r_{xy}=\frac{s_{xy}}{s_x s_y}\in[-1,1]
\end{equation*}
where $s_x,s_y$ are the empirical st.dev.\

\textbf{Remarks}:
\begin{itemize}
    \item Correlation only measures linear relations
    \item the sign of $r_{xy}$ measures the direction and $|r_{xy}|$ the strength of the linear relation.
    \item if $|r_{xy}|=1$, then the dependence is deterministic
\end{itemize}
\begin{center}
    \includegraphics[width =\linewidth]{../python/pdf_plots/correlation.pdf}
\end{center}