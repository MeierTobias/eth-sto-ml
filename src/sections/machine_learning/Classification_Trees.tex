\section{Classification Trees}
A classification tree
\noindent\begin{equation*}
    t: \mathbb{R}^2 \to \{1,\ldots, M\}
\end{equation*} consists of
\begin{itemize}
    \item a partition of $\mathbb{R}^d$:
          \noindent\begin{equation*}
              S_1\cup \cdots S_K = \mathbb{R}^d \;:\; S_i\cap S_j=\emptyset , i,j\leq K
          \end{equation*}
    \item An assignment of lables to each partition from a predefined set of classes $\{1,\ldots, M\}$.
    \item A dataset of labeled examples is given by
          \noindent\begin{equation*}
              D=\{(x_{1},y_{1}),\ldots,(x_{n},y_{n})\}\subseteq\mathbb{R}^{d\times M},\qquad x\in \mathbb{R}^d
          \end{equation*}
\end{itemize}

\subsection{Entropy}
One way to measure the quality of a tree is by calculating the entropy resulting form the partition:
\noindent\begin{equation*}
    H(D)=-\sum_{j\leq M}p_j\log p_j
\end{equation*}

In the discrete case:
\noindent\begin{equation*}
    H(X)=-\sum_{x\in X}\mathbb{P}(X=x)\log\mathbb{P}(X=x)
\end{equation*}
and in the continous case:
\noindent\begin{equation*}
    H(X)=-\int p(x)\log p(x) dx
\end{equation*}

As a result, $t^*$ is the tree that optimizes the loss $L$
\noindent\begin{equation*}
    t^* = \min_t L(D,t) =\min_t \frac{1}{K}\sum_{i=1}^{K} \frac{|D_i|}{|D|}H(D_i)
\end{equation*}

\subsection{Greedy Algorithm}

\begin{lstlisting}[style=bright_C++]
def train(D):
    if H(D) = 0 then:
        return  y, where y is the only label 
        appearing in D. 
    else:
        compute the partitioning line (f,t) 
        that minimizes Q(D,f,t) . 
        let D_l and D_r be the parts resulting from (f,t)
        return (f>=t)? train(D_l) : train(D_r)
\end{lstlisting}