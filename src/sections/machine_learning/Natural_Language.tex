\section{Natural Language Processing}
\subsection{Preprocessing}
\subsubsection{Tokenization}
Tokenization is used to split up words or other input data into smaller \textbf{tokens}. This ensures that the inputs are more regular in size and introduces inter-word context.

\subsection{Embeddings}

\subsubsection{Bag of Words}
BoW is a simple way of representing text, by counting the number of occurences of the $n-1$ most frequent words (all other words are assigned `UNK').
This results in a $n$ dimensional vector representation of the text sequence.
\begin{itemize}
    \item [+] Simple
    \item [-] No notion of dependencies between words (focuses only on frequency of words)
    \item [-] Sparsity 
\end{itemize}
\subsubsection{Word Embeddings}
In contrast to BoW \textit{word embeddings} take the context of the word/token into account (but \textbf{not} their sequence).
\begin{itemize}
    \item Similar words have similar embeddings (distributional hypothesis) as one takes into accout the surrounding.
    \item The general strategy of Continous BoW (CBOW) and skip-gram is to extract embeddings as a by-product of training a NN:
    \begin{enumerate}
        \item train a NN for a prediction task (predict gap or surrounding of a word)
        \item use some of the trained layer weights as dense word embeddings
    \end{enumerate}
    \item The main components in CBOW and skip-gram are:
    \begin{itemize}
        \item $\mathbf{x}_w \in \mathbb{R}^d$: embedding of target word
        \item $\mathbf{z} _w$ in $\mathbb{R}^d$: embedding of given word(s) (summed up for CBOW)
        \item $\mathcal{V}$: Vocabulary
        \item $d$: embedding dimension (usually $\in [50\dots500]$)
    \end{itemize}
\end{itemize}

\paragraph{CBoW}
\begin{itemize}
    \item context $\to$ word
    \item faster and better for frequent words
\end{itemize}

The functional one wants to maximize is:
\noindent\begin{gather*}
    J_\theta^{\mathsf{CBOW}}                                               = \sum_{t}\log\left(p(w_t|w_{t-c},\ldots, w_{t-1},w_{t+1},\ldots, w_{t+c})\right)                                   \\
    p(\underbrace{v}_{\textsf{target}} | \underbrace{w}_{\textsf{given}})  = \frac{\exp(\mathbf{x}_v^{\mathsf{T}}\mathbf{z}_w)}{\sum\limits_{u\in \mathcal{V}} \mathbf{x}_u^{\mathsf{T}}\sum\mathbf{z}_w} \quad \mathcal{V}: \text{Vocabulary}
\end{gather*}

\newpar{}
\ptitle{Example Calculation}
\begin{center}
    \includegraphics[width=\linewidth]{nlp_cbow.png}
\end{center}
\begin{enumerate}
    \item create one-hot encoding of inputs
	\item multiply each word by $W_1$ and sum the results 
	\item multiply the sum by $W_2$
	\item apply softmax and round to get one-hot encoding of gap
\end{enumerate}

\paragraph{Skip-Gram}
\begin{itemize}
    \item Word $\to$ Context
    \item better for smaller datasets and infrequent words
\end{itemize}

One wants to maximize the functional
\noindent\begin{gather*}
    J_{\theta}^{\mathsf{SG}} = \sum_{t}\sum_{\overset{l=-c}{l\neq 0}}^{c} \log(p(w_{t+l}|w_t))\\
    p(\underbrace{v}_{\textsf{target}} | \underbrace{w}_{\textsf{given}})  = \frac{\exp(\mathbf{x}_v^{\mathsf{T}}\mathbf{z}_w)}{\sum\limits_{u\in \mathcal{V}} \mathbf{x}_u^{\mathsf{T}}\mathbf{z}_w} \quad \mathcal{V}: \text{Vocabulary}
\end{gather*}

\newpar{}
\ptitle{Example Calculation}
\begin{center}
    \includegraphics[width=\linewidth]{nlp_skip.png}
\end{center}
The embedding of a one-hot encoded word can then be obtained by multiplication with the trained weight matrix $W_1$.

\subsubsection{Sequences of Words}
\ptitle{n-Grams}
N-grams are $n$ consecutive words/tokens in a text.
\begin{itemize}
    \item \textbf{Unigrams} treat each word independently
    \item \textbf{Bigram}, \textbf{trigrams} etc.\ treat 2,3,$\ldots$ words in combination
\end{itemize}
\textbf{Remarks}
\begin{itemize}
    \item The combination of words grows with $\mathcal{O}(c^n)$
    \item An \textbf{n-gram language model} predicts the probability of word $n$ given the words $0,\dots,n-1$. Two of the main issues are:
    \begin{itemize}
        \item Sparsity: sentences that are rare in the corpus have $P=0$ to be predicted. For sentences not given in the corpus one can't calculate $P$ at all.
        \item Storage: if the corpus is large storing all the possbile word permutations becomes hard.
    \end{itemize}
\end{itemize}

\newpar{}
See RNN Section~\ref{sec:RNN}

\subsubsection{Position Encoding}
Through position encoding, information about the position of a word/token can be incorporated. As a result, the same word/token can have a differnet representation/embedding at different positions.
\newpar{}
\ptitle{Sine/Cosine Positional Encoding}

Usually one uses sine and cosine functions to add positional information to the encoding.
\begin{align*}
    PE_{(pos,2i)}&=\sin\left(\frac{pos}{10000^{\frac{2i}{d_{model}}}}\right)\\
    PE_{(pos,2i+1)}&=\cos\left(\frac{pos}{10000^{\frac{2i}{d_{model}}}}\right)
\end{align*}
Remarks:
\begin{itemize}
    \item sine is used for even positions, cosine for odd occurences
    \item $d_{model}$ is the number of entries in the embedding vector
    \item $pos$ is the word's position in the sentence
    \item overall-embedding of a word is given by $E+PE$ where $E$ is the common word embedding
    \item two different words have the same $PE$ if they reside at the same position within their sentences
\end{itemize}

\subsection{Language Models}
\subsubsection{Attention}
\textit{Attention is a fuzzy, differentiable, vectorized dictionary look-up.}

\newpar{}
Attention enhances the performance by enabling long-range dependencies without sequential processing.

\newpar{}
A attention block takes
\begin{itemize}
    \item \textbf{Query}: The word to translate
    \item \textbf{Key}: The word in source languange
    \item \textbf{Value}: The word in target language
\end{itemize}
as inputs.

\newpar{}
\ptitle{Scaled Dot-Product Attention\;\; Multi-Head Attention}
\begin{center}
    \includegraphics[width=\linewidth]{nlp_attention.png}
\end{center}

\begin{examplesection}[Calculating Attention]
    Given
    \noindent\begin{equation*}
        \mathbf{X}\in \mathbb{R}^{n_{\mathsf{words}} \times d_{\mathsf{rep}}}\qquad (\mathbf{W}_Q, \mathbf{W}_K, \mathbf{W}_V)\in \mathbb{R}^{d_{\mathsf{rep}}\times n_{\mathsf{words}}}
    \end{equation*}
    Scaled dot-product attention is given by
    \noindent\begin{equation*}
        \mathrm{softmax}\left(\frac{\mathbf{QK}^{\mathsf{T}}}{\sqrt{d_{\mathsf{rep}}}}\right) \mathbf{V},\qquad
        \begin{cases}
            \mathbf{Q} = \mathbf{XW}_Q \\
            \mathbf{K} = \mathbf{XW}_K \\
            \mathbf{V} = \mathbf{XW}_V
        \end{cases}
    \end{equation*}
\end{examplesection}

\paragraph{Self- and Cross-Attention}

\ptitle{Self-Attention}:

Words can correspond to different words within the same sequence.
\newpar{}
\ptitle{Cross-Attention}:

Words can have different meaning between different sequences:

\begin{center}
    \includegraphics[width=.3\linewidth]{nlp_cross_attention.png}
\end{center}

\subsubsection{Transformers}
\newpar{}
\ptitle{Key Properties}
\begin{itemize}
    \item a type of NN architecture
    \item no exploding or vanishing gradients
    \item can be trained in parallel
    \item good at handling long-range dependencies in text and images (enabled through attention blocks)
\end{itemize}
\paragraph{Basic Architecture}
\begin{center}
    \includegraphics[width=\linewidth]{nlp_transformer.png}
\end{center}

\newpar{}
\ptitle{Remarks}
\begin{itemize}
    \item Residual connections are used to mitigate vanishing gradients.
\end{itemize}

\paragraph{Attention in Transformers}
\newpar{}
\ptitle{High-Level Function of Attention Blocks}
Attention blocks
\begin{itemize}
    \item give the decoder access to the entire input 
    \item provide weights for the decoder to decide on which input word is how important for the next output
    \item calculate what one needs to add to a generic embedding to get an embedding representing context
    \item pass information from many words' embeddings to the embedding of one word
\end{itemize}

\newpar{}
\ptitle{Nomenclature}
\begin{itemize}
    \item \textit{Initial embedding}: encodes a word just by itself and a positional encoding, ignoring context 
    \item \textit{Refined embedding}: embedding encoding context information
    \item \textit{Attention heads}: update the initial embedding by a certain relation (e.g.\ how adjectives specify the meaning of a noun more precisely) to take a step towards the final refined encoding
    \item \textit{Query}: encodes a certain ``question'' (``What did the cat catch''?) in a lower dimension than the embedding vector
\end{itemize}
 
\newpar{}
\ptitle{Creating Attention}
\begin{enumerate}
    \item 
\end{enumerate}

\paragraph{Encoder}
In the encoding part, information in the text sequence is \textit{encoded} into \textbf{representation vectors}.
\newpar{}
\ptitle{Training}
\begin{itemize}
    \item Use the complete sequence for training
    \item Trained based on word masking (self-supervised):
          \begin{itemize}
              \item Replace some words with \fncode{<mask>} and some with random words
              \item Model needs to find the masked/replaced words
          \end{itemize}
\end{itemize}
\begin{center}
    \includegraphics[width=.4\linewidth]{nlp_enc_training.png}
\end{center}


\paragraph{Decoder}

Based on the representations of the encoder, the decoder generates output from the \textbf{representation vectors}
\newpar{}
\ptitle{Training}
\begin{itemize}
    \item Only past sequence matters (causal attention)
    \item Training based on \textit{next word prediction} (self-supervised)
\end{itemize}
\begin{center}
    \includegraphics[width=.4\linewidth]{nlp_dec_training.png}
\end{center}

\subsubsection{Multimodel Models}
Transformers are not limited to text sequences, tokens can also be generated from audio, images or other sources and their combination.
