\section{Linear Regression}
Linear regression is a form of \textbf{statistical learning}.
\subsection{Simple Linear Regression (SLR)}
For SLR there is only one feature in contrast to MLR.

\subsubsection{Dataset}
The set of pairs
\begin{align*}
    (x,y) & \in\mathbb{R}\times\mathbb{R}       \\
    x:    & \text{ input variable, regressor}   \\
    y:    & \text{ output variable, regressand}
\end{align*}
forms the dataset for SLR
\begin{equation*}
    D=\{(x_1,y_1),(x_2,y_2),\dots,(x_n,y_n)\}
\end{equation*}

\subsubsection{Model}
Assuming $D$ is a realization of a sample
\begin{equation*}
    (X_1,Y_1),(X_2,Y_2),\dots,(X_n,Y_n)\thicksim F
\end{equation*}
SLR assumes for the \textbf{model} $\mathcal{H}$ the distribution $F$ as
\begin{align*}
    Y_i        & =\beta_0X_i+\beta^{\prime}_0+\epsilon_i \\
    \epsilon_i & \sim N(0,\sigma^2): \text{ noise}
\end{align*}
where $\beta_0,\beta^{\prime}_0,\sigma\in\mathbb{R}$ are unknown and $\mathcal{H}$ is the set of all functions of the form
\begin{equation*}
    f(x)=\beta x+\beta'
\end{equation*}

\ptitle{Justification}

Justification for this choice is given by the assumption that the output variable $y$ is proportional to the input variable $x$ plus a base value plus some noise.

\subsubsection{Loss Function}
Given the dataset $D$ and the \textbf{estimator} $f$ we choose the \textbf{mean squared error (MSE)}
\begin{equation*}
    L(D,f)=\frac{1}{n}\sum_{i\leq n}{\left(y_i-f(x_i)\right)}^2
\end{equation*}
as cost function.

\ptitle{Remarks}

\begin{itemize}
    \item Another natural choice would have been $|f(x)-y|$ which is not differentiable.
    \item MSE punishes large deviations more than small ones.
\end{itemize}

\subsubsection{Training Algorithm}
One tries to get the best estimator for $\hat{f}(x)=\hat{\beta}x+\widehat{\beta^{\prime}}$ by finding
\begin{equation*}
    \hat{f}=argmin_{f\in\mathcal{H}}L(D,f)
\end{equation*}

\ptitle{Optimal Parameters}

The parameters minimizing the MSE are given by
\begin{equation*}
    \begin{pmatrix}
        \hat{\beta} \\
        \widehat{\beta'}
    \end{pmatrix}
    ={(\mathbf{X}^T \mathbf{X})}^{-1}\mathbf{X}^T\mathbf{\gamma}
\end{equation*}
where
\begin{equation*}
    \mathbf{X}=
    \begin{pmatrix}
        x_1 & 1 \\
        x_2 & 1 \\
        \vdots  \\
        x_n & 1
    \end{pmatrix},\quad
    \mathbf{\gamma}=
    \begin{pmatrix}
        y_1    \\
        y_2    \\
        \vdots \\
        y_n
    \end{pmatrix}
\end{equation*}

\paragraph{Statistical Testing and Confidence Intervals}

\ptitle{Confidence Intervals}

$\hat{\beta}$ is a statistical estimator of $\beta$. Therefore, one can derive a confidence interval on it.

\ptitle{p-Values}

\begin{itemize}
    \item One can also test with $p$-values.
    \item E.g. $H_0$: $\beta_0=0$. The null hypothesis assumes that there is no correlation between area and house prices.
    \item A low $p$-value indicates a correlation.
\end{itemize}

\ptitle{Remarks}

\begin{itemize}
    \item $\hat{f}$ is called the \textbf{ordinary least-squares estimator (OSLE)}
\end{itemize}

\paragraph[Validation with R2]{Validation with $R^2$ score}
\begin{equation*}
    R^2(D,\hat{f})=1-\frac{MSE(D,\hat{f})}{MSE(D,f_0)}\in\left[-\infty,1\right)
\end{equation*}

with the \textbf{dummy estimator}
\begin{equation*}
    f_0=\frac1n\sum_{i}{y_i}
\end{equation*}

\ptitle{Remarks}

\begin{itemize}
    \item Comparing the MSE on the test data is not informative.
    \item Better: The $R^2$ score compares the estimator's performance with the performance of a dummy estimator.
    \item One tries to get as close as possible to $R^2=1$
\end{itemize}

\subsection{Multivariate Linear Regression (MLR)}
\subsubsection{Dataset}
Similar to SLR but now the input variable is a vector containing all features i.e.

\begin{align*}
    (\mathbf{x},y) & \in\mathbb{R}^d\times\mathbb{R}     \\
    \mathbf{x}:    & \text{ input variable, regressor}   \\
    y:             & \text{ output variable, regressand}
\end{align*}
and
\begin{equation*}
    D=\{(\mathbf{x_1},y_1),(\mathbf{x_2},y_2),\dots,(\mathbf{x_n},y_n)\}
\end{equation*}

\subsubsection{Model}
One assumes $D$ is a realization of a sample
\begin{equation*}
    (\mathbf{X}_1,Y_1),(\mathbf{X}_2,Y_2),\dots,(\mathbf{X}_n,Y_n)\thicksim F
\end{equation*}
and
\begin{align*}
    Y_i        & =\boldsymbol{\beta}_0\mathbf{X}_i+\beta^{\prime}_0+\epsilon_i \\
    \epsilon_i & \sim N(0,\sigma^2): \text{ noise}
\end{align*}
where $\boldsymbol{\beta}_0\in \mathbb{R}^d$, $\beta^{\prime}_0,\sigma \in \mathbb{R}$.

\subsubsection{Loss Function}
\begin{equation*}
    L(D,f)=\frac{1}{n}\sum_{i\leq n}{\left(y_i-f(\mathbf{x_i})\right)}^2
\end{equation*}


\subsubsection{Training Algorithm}
Finds the best estimator for
\begin{align*}
    \hat{f}(x) & =\hat{\boldsymbol{\beta}}^T x+\widehat{\beta^{\prime}} \\
               & =argmin_{f\in\mathcal{H}}L(D,f)
\end{align*}

\ptitle{Optimal Parameters}

The parameters minimizing the MSE are given by
\begin{equation*}
    \begin{pmatrix}
        \hat{\boldsymbol{\beta}} \\
        \widehat{\beta'}
    \end{pmatrix}
    ={(\mathbf{X}^T \mathbf{X})}^{-1}\mathbf{X}^T\mathbf{\gamma}
\end{equation*}
with
\begin{equation*}
    \mathbf{X}=
    \begin{pmatrix}
        {\mathbf{x}_1}^T & 1 \\
        {\mathbf{x}_2}^T & 1 \\
        \vdots                  \\
        {\mathbf{x}_n}^T & 1
    \end{pmatrix},\quad
    \mathbf{\gamma}=
    \begin{pmatrix}
        y_1    \\
        y_2    \\
        \vdots \\
        y_n
    \end{pmatrix}
\end{equation*}
where $\mathbf{X}$ is called \textbf{design matrix}.

\subsubsection{Statistical Testing and Confidence Intervals}
Similar to SLR one can perform tests using $p$-values or rate the estimator using confidence intervals.

\subsection{Practical Considerations}

\subsubsection{Outliers}
\begin{itemize}
    \item Linear regression is strongly affected by outliers.
    \item Data should be visualized (e.g.\ by 2D scatter plots) to detect outliers.
\end{itemize}

\subsubsection{Feature Engineering}

A transformation is a function
\begin{equation*}
    \varphi{:}\:\mathbb{R}^d\to\mathbb{R}^m
\end{equation*}

\begin{itemize}
    \item Input variables can be transformed as they are not set to have a linear dependence with the output variable.
    \item A transformation can create multiple features (i.e.\ functions of $x_i$) from one feature $x_i$.
    \item In this case $\varphi$ creates MLR from SLR.
\end{itemize}

\ptitle{Procedure}

\begin{enumerate}
    \item Transform dataset $D$ into $E$
    \item Train linear regression estimator $f*$ on $E$
    \item Estimate a new $y$ by computing $f(\varphi(x))$
\end{enumerate}


\ptitle{Example}

\begin{align*}
    y_i & =\beta_0+\beta^{\prime}{}_0x_i+\beta^{\prime\prime}{}_0x_i^2+\beta^{\prime\prime\prime}{}_0x_i^3+\beta_0^{(4)}\sqrt{x_i}+\beta_0^{(5)}e^{x_i}+\epsilon_i \\
    y_i & ={\left(\beta_0,\beta^{\prime}{}_0,\dots,\beta_0^{(5)}\right)}^{\mathsf{T}}\left(1,x_i,x_i^2,x_i^3,\sqrt{x_i},e^{x_i}\right)+\epsilon_i                  \\
    y_i & =\overline{\beta_0}^T\varphi(x_i)+\epsilon_i
\end{align*}
with $\overline{\boldsymbol{\beta}_0}=\begin{pmatrix}\beta_0,\beta_0',\beta_0'',\beta'''_0,\beta_0^{(4)},\beta_0^{(5)}\end{pmatrix}$. Therefore one created MLR from SLR.

\paragraph{Kernelization}
\begin{itemize}
    \item The optimal transformation would contain any possible function of the input variables ($\infty$-dimensional).
    \item Kernels allow to encode some types of $\infty$-dimensional transformations.
\end{itemize}

\ptitle{Radial Bias Function (RBF) Kernel}

RBF encodes a transformation $\varphi$ containing all possible products that can be formed with all features in $\mathbf{x}$ with
\begin{equation*}
    {\psi(\mathbf{x})}^T\psi(\mathbf{x}')=\exp\left(-\frac{\left\|\mathbf{x}-\mathbf{x}'\right\|^{2}}{\gamma}\right)
\end{equation*}
with $\gamma \in \mathbb{R}$.

\ptitle{Support-Vector Regression (SVR)}

\begin{itemize}
    \item A model that uses transformations to fit curves to data.
    \item Implemented in scikit-learn in the \textbf{SVR class}.
\end{itemize}


\subsubsection{Multicollinearity}

\subsubsection{Standardization}

