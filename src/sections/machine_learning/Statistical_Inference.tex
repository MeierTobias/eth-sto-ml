\newcol{}
\section{Statistical Inference}
Statistical inference can be split into
\begin{itemize}
    \item \textbf{Frequentism}:
          \begin{enumerate}
              \item Pick a parametric model
              \item Fit the model using MLE to obtain a best parameter
          \end{enumerate}
          \newpar{}
          \begin{itemize}
              \item[+] Tractable (just differentiation of $\log(L)$)
              \item[+] Asymptotically unbiased
              \item[-] Stability and variance issues
          \end{itemize}
          \newpar{}
    \item \textbf{Bayesianism}:
          \begin{enumerate}
              \item Guess a prior model of the parameter
              \item Pick a parametric model
              \item Update the prior with the parametric model to obtain a posterior \textbf{model}
          \end{enumerate}
          \newpar{}
          \begin{itemize}
              \item[+] Low variance
              \item[-] Intractable
              \item[-] Bias issues
          \end{itemize}
    \item \textbf{Non-Parametric Statistics}:
          \begin{itemize}
              \item Frequentism/Bayesianism with non-parametric model
          \end{itemize}
\end{itemize}

\subsection{Likelihood}
The likelihood is the probabilty of observing $X$ with i.i.d.\ $X_i$ given that $f$ is parametrized by $\theta$:
\noindent\begin{equation*}
    L(\theta|X) = p(X|\theta) = \prod_{i=1}^{n} f(X_i|\theta)
\end{equation*}

\subsection{Frequentism}
By choosing a model $\mathcal{H}$, one can introduce prior knowledge.

\subsubsection{Parametric Model}
Assuming $X\sim F$ where $F$ is any distribution, the goal is to find parameters $\theta\in\Theta$ such that the model $\mathcal{H}$ ``best fits'' the samples $X$.

\begin{examplesection}[Example]
    \noindent\begin{equation*}
        \mathcal{H}=\Bigl\{\mathcal{N}(\theta, \sigma^2): \theta\in\Theta\Bigr\}\qquad \begin{cases}
            \Theta = [40,100] \\
            \sigma^2 = 10
        \end{cases}
    \end{equation*}
    or in the multivariate case
    \noindent\begin{equation*}
        \mathcal{H}=\Bigl\{\mathcal{N}(\mu, \Sigma): \mu\in\Theta_\mu, \Sigma\in\Theta_\Sigma\Bigr\}\quad \begin{cases}
            \Theta_\mu\subseteq \mathbb{R}^{d+1} \\
            \Theta_\Sigma\subseteq \mathbb{R}^{(d+1)\times (d+1)}
        \end{cases}
    \end{equation*}
    where the covariance matrix $\Theta_\Sigma$ is a subset of positive definite matrices. Note however, that the problem is mathematically demanding and the simplification shown under~\ref{MLE_simp} can be made.
\end{examplesection}


\subsubsection{MLE}
The \textit{maximum likelihood estimator} $\theta^*_{MLE}$ is given by
\noindent\begin{align*}
    \theta^*_{MLE} & = \arg\max_{\theta\in\Theta} L(\theta|X) =   \arg\max_{\theta\in\Theta} p(X|\theta)                         \\
                   & = \arg\max_{\theta\in\Theta} \log(p(X|\theta)) \overset{X\sim\mathcal{N}}{=} \frac{1}{n} \sum_{i=1}^{n} x_i
\end{align*}

\textbf{Remark}
\begin{itemize}
    \item By the law of large numbers if $X\sim \mathcal{N}(\theta_0, \sigma^2)$, $\theta^*_{MLE}\to\theta_0$ as $n\to\infty$
\end{itemize}

\begin{examplesection}[House Price Model]\label{MLE_simp}
    Assuming given sample data
    \begin{equation*}
        Z=\{(x_1,y_1),\dots,(x_n,y_n)\} \sim F
    \end{equation*}
    Then, if the model
    \noindent\begin{align*}
        y_i|x_i             & \sim \mathcal{N}({\beta_0}^{\mathsf{T}}\mathbf{x}_i, \sigma^2)\quad \sigma>0, \beta_0\in \mathbb{R}^d, i\leq n \\
        p(Z|\beta,\sigma^2) & = \prod\limits_{i=1}^n p(\mathbf{x}_i,y_i|\beta,\sigma^2)                                                      \\
                            & \propto \prod\limits_{i=1}^n p(y_i|\beta,\sigma^2)                                                             \\
                            & \vdots                                                                                                         \\
        p(Z|\beta)          & \propto \prod\limits_{i=1}^n \exp\left(-y_i -\beta^{\mathsf{T}}\mathbf{x}_i\right)
    \end{align*}
    is chosen, the problem is reduced from fitting a multivariate Gaussian to a univariate Gaussian. The \textbf{MLE} is then equal to the \textbf{OLSE}:
    \noindent\begin{align*}
        \log(p(Z|\beta))                    & = C -\frac{1}{{\sigma_0}^2}\sum_{i=1}^{n} {(y_i-\beta^{\mathsf{T}}\mathbf{x}_i)}^2          \\
                                            & =C - \frac{1}{{\sigma_0}^2}L(\beta,Z)                                                       \\
        \underbrace{\beta^*}_{\textsf{MLE}} & = \arg\max_\beta \log(p(Z|\beta)) = \underbrace{\arg\min_\beta L(\beta, Z)}_{\textsf{OLSE}}
    \end{align*}
\end{examplesection}

\subsubsection{MAP}
Some issues arise if linear models generalize to MLE (overfitting in case of many features and/or few samples, numerical instability).
As a solution, the \textit{maximum a priori} estimator $\theta^*_{MAP}$ is similar to the MLE but with an added prior guess of an underlying distribution of $\theta$, according to \textit{Bayes' law}:
\noindent\begin{align*}
    \theta^*_{MAP} & = \arg\max_{\theta\in\Theta} P(\theta|X)  = \arg\max_{\theta\in\Theta} \frac{p(x|\theta)p(\theta)}{\int p(x|\theta)p(\theta)\, d\theta} \\
                   & = \arg\max_{\theta\in\Theta} p(X|\theta)p(\theta)                                                                                       \\
                   & = \arg\max_{\theta\in\Theta}\left(\log(p(\theta)) + \sum_{i=1}^{n}\log(f(X_i|\theta))\right)
\end{align*}

\subsection{Bayesianism}
Bayesianism uses the same argument as MAP but estimates the entire distribution instead of just $\Theta$.

\subsubsection{Bias-Variance Decomposition}
For some models, increasing the bias of a model results in a significant reduction of variance of an estimator and as a result the overall squared error is reduced:
\noindent\begin{align*}
    \mathbb{E}_{Z,\mathbf{x},y}\Bigl[{(y-f_Z(\mathbf{x}))}^2\Bigr] & = \Biggl(\overbrace{\mathbb{E}_{\mathbf{x}}\Bigl[\mathbb{E}_{Z}\bigl[f_Z(\mathbf{x})\bigr]-\mathbb{E}_{y}\bigl[y\bigr]\Bigr]}^{\textsf{Bias of model}}\Biggr)^2 + \\%Chktex 3
                                                                   & +\underbrace{\mathbb{E}_{Z,\mathbf{x}}\Bigl[{\Bigl(\mathbb{E}_{Z}\bigl[f_Z(\mathbf{x})\bigr]-f_Z(\mathbf{x})\Bigr)}^2\Bigr]}_{\textsf{Variance of algorithms/estimator}}+    \\
                                                                   & +\underbrace{\mathbb{E}_{y}\Bigl[{\Bigl(y-\mathbb{E}_{y}\bigl[y\bigr]\Bigr)}\Bigr]}_{\textsf{Noise}}
\end{align*}

\subsubsection{Steps}
\begin{enumerate}
    \item Choose a \textbf{prior} distribution of the parameter $\theta$ (before obtaining data):
          \noindent\begin{equation*}
              p(\theta)
          \end{equation*}
    \item Compute the \textbf{likelihood} distibution of the sample $Z=(X,Y)$ given the parameter $\theta$:
          \noindent\begin{equation*}
              p(X|\theta) = p(x_1,x_2,\ldots,x_n|\theta)
          \end{equation*}
    \item Compute a corrected \textbf{posterior} distribution over $\theta$:
          \noindent\begin{align*}
              p(\theta|X) & =\frac{p(X|\theta)p(\theta)}{p(X)}      =\frac{p(X|\theta)p(\theta)}{\int p(X|\theta)p(\theta) \, d\theta} \\
                          & \propto p(X|\theta)p(\theta)
          \end{align*}
\end{enumerate}

\newpar{}
\ptitle{Remarks}
\begin{itemize}
    \item In contrast to MLE/MAP (yield only $\Theta$) we now have $p(\theta|X)$ describing uncertainty in our parameter estimate.
    \begin{itemize}
        \item One can update it with new data.
    \end{itemize}
\end{itemize}

\begin{examplesection}[Example]
    Assuming a model depending on $\alpha, \beta$
    \noindent\begin{equation*}
        p(\theta)    =  \mathcal{N}(\alpha,\beta^2)                                                                                                                                                                                      
    \end{equation*}
    One can show that the posterior distribution is given by
    \begin{equation*}
        p(\theta|X)  =\mathcal{N}\left({\left(\frac{1}{\beta^{2}}+\frac{n}{\sigma^{2}}\right)}^{-1}{\left(\frac{\alpha}{\beta^{2}}+\frac{n}{\sigma^{2}}\bar{X}\right),\left(\frac{1}{\beta^{2}}+\frac{n}{\sigma^{2}}\right)}^{-1}\right)
    \end{equation*}
    which for $n=0$ is the prior, for $n\rightarrow \infty$ concentrates around $\bar{X}$. 
\end{examplesection}

\begin{examplesection}[House Prices]
    \textbf{Sample Data}:
    \begin{equation*}
        Z=\{(x_1,y_1),\dots,(x_n,y_n)\} \sim F
    \end{equation*}
    \textbf{Model}:
    \noindent\begin{equation*}
        y_i  \sim \mathcal{N}({\beta_0}^{\mathsf{T}}\mathbf{x}_i, \sigma^2)\quad \sigma>0, \beta_0\in \mathbb{R}^d, i\leq n \\
    \end{equation*}
    \textbf{Prior}:
    \noindent\begin{equation*}
        \beta_0\sim\mathcal{N}(0,s_0^2 \mathbf{I}    )
    \end{equation*}
    \textbf{Likelihood}:
    \noindent\begin{equation*}
        p(Z|\beta)\propto\exp\left(-\frac{1}{\sigma^{2}}\sum_{i\leq n}{(y_{i}-\beta^{\top}x_{i})}^{2}\right)
    \end{equation*}
    \textbf{Posterior}:
    \noindent\begin{equation*}
        p(\beta|Z)=\mathcal{N}\left({{\left(\frac{\sigma^{2}}{s_{0}^{2}}I+X^{\mathsf{T}}X\right)}^{-1}X^{\mathsf{T}}y,\;\frac{1}{2({\frac{1}{s_{0}^{2}}I+\frac{1}{\sigma^{2}}X^{\mathsf{T}}X})}}\right)
    \end{equation*}

    \textbf{Ridge Estimator}:

    In this case, the parameter $\beta$ takes its maximum in
    \noindent\begin{equation*}
        {\left(\frac{\sigma^{2}}{s_{0}^{2}}I+X^{\mathsf{T}}X\right)}^{-1}X^{\mathsf{T}}y
    \end{equation*}
    i.e.\ the ridge estimator is the result of conducting Bayesian inference on a linear model, where the prior believes that all coefficients are 0.
\end{examplesection}