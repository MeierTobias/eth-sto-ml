\section{Clustering}

\subsection{K-Means}
K-Means is a type of unsupervised learning, where data is grouped into $K$ \textit{clusters}.
The \textit{Hypothesis space}
\noindent\begin{equation*}
    \mathcal{H} = \{(\boldsymbol{\mu}_1,\ldots,\boldsymbol{\mu}_K,\mathbf{c})\}\qquad \forall k=1,\ldots, K
\end{equation*}
consists of \textbf{cluster centers} $\boldsymbol{\mu}_k\in \mathbb{R}^d$ and \textbf{cluster assignments} $\mathbf{c} = (c_1,\ldots,c_N)\in\{1,\ldots, K\}$

\newpar{}
The loss function
\noindent\begin{equation*}
    \mathcal{L}=\sum_{i=1}^{N} \|\mathbf{x}_i-\boldsymbol{\mu}_{c_i} \|_2^2
\end{equation*}
is given by the quadratic distance to the assigned center.

\textbf{Remark}
Standardization of the dataset $D$ is important.

\subsubsection{Expectation-Maximization Algorithms}
EM-Algorithms find a set of local MLEs $\theta^*$ by repeating two steps:
\begin{enumerate}
    \item \textbf{E-Step}: use old $\theta$ to find complete data posterior distribution of latent variable $\mathbf{c}$
    \item \textbf{M-Step}: use updated $\mathbf{c}$ to find $\theta_{new}$ that \textbf{maximizes} the complete-data posterior likelyhood
\end{enumerate}

\paragraph{Lloyd's Algorithm}
\ptitle{\code{Lloyd($K>0$, Dataset $D$)}}
\fncode{
    \begin{algorithmic}
        \setstretch{1.5}
        \State{$\boldsymbol{\mu}_1, \ldots,\boldsymbol{\mu}_K \gets$ random cluster centers}
        \While{not converged}
        \ForAll{$i$}
        \State{Update assignments: $c_i\gets \arg\min\limits_{k} \|\mathbf{x}_i-\boldsymbol{\mu}_k\|$}
        \EndFor{}
        \ForAll{$k$}
        \State{Update cluster centers: $\boldsymbol{\mu}_k\gets \frac{\sum_{i:c_i=k}\|\mathbf{x}_i-\boldsymbol{\mu}_k\|}{|\{i:c_i=k\}|}$}
        \State{}\Comment{$i:c_i=k$: data points assigned to cluster $k$}
        \EndFor{}
        \EndWhile{}
    \end{algorithmic}
}

\subsubsection{Number of Clusters - Elbow Criterion}
The sums of square errors follow the rule
\noindent\begin{equation*}
    \underbrace{\sum_{i=1}^{n}\|\mathbf{x}_i-\bar{\boldsymbol{\mu}}\|^2}_{\textsf{Total: TSS}} = \underbrace{\sum_{i=1}^{n}\|\mathbf{x}_i-\boldsymbol{\mu}_{c_i}\|^2}_{\textsf{Within: SSW}} + \underbrace{\sum_{i=1}^{n}\|\boldsymbol{\mu}_{c_i}-\bar{\boldsymbol{\mu}}\|^2}_{\textsf{Between: SSB}}
\end{equation*}

The \textbf{Elbow criterion} states that $K$ should be chosen such that an additional cluster $(K+1)$ yields little improvement in the SSW.

\ptitle{Alternative Approaches}
\begin{itemize}
    \item \textbf{Constraint optimization}: Add a penalty term for a large number of clusters to the cost function.
    \item \textbf{Stability-based clustering}: clusters should represent novel data points.
    \item \textbf{Given by application}
\end{itemize}