\section{Sample Spaces and Probability Measures}
\subsection{Notation}
\noindent\begin{align*}
     & \omega               &  & \text{Possible outcome}                               \\
     & \Omega               &  & \text{Sample space}                                   \\
     & A \subseteq \Omega   &  & \text{Event}                                          \\
     & \{x_1 \dots x_n \}^k &  & \text{all sequences of length k using elements } x_i. \\
\end{align*}
\subsection{De Morgan's Laws}
\noindent\begin{align*}
    {(A\cup B)}^C = A^C\cap B^C \\
    {(A\cap B)}^C = A^C\cup B^C
\end{align*}

\subsection{Axioms of Probability Theory}
\begin{enumerate}
    \item $0\leq \mathbb{P}(A)\leq 1$
    \item $\mathbb{P}(\Omega)$ = 1
    \item $\mathbb{P}\left(\cup_{i\geq 1} A_i\right) = \mathbb{P}\underbrace{(A_1 \cup A_2 \cup \dots)}_{\text{countably infinite}} = \sum_{i\geq 1} \mathbb{P}(A_i)$\\
          if $A_{i} \cap A_{j} = \emptyset \; \forall i \ne j$
\end{enumerate}
A sample space $\Omega$ with a probability measure $\mathbb{P}$ forms a \textbf{probability space}.

\subsubsection{Further rules from the Axioms}\label{sssec:rules_from_axioms}
\noindent\begin{align*}
     & \mathbb{P}(\emptyset) = 0                                                                                                                                  \\
     & \mathbb{P}  \underbrace{(A_1 \cup \dots \cup A_n)}_{\text{finite}} = \sum_{i=1}^{n} \mathbb{P}(A_i) & \text{if } A_i\cap A_j = \emptyset\, \forall i\neq j \\
     & \mathbb{P}(A^C) = 1-\mathbb{P}(A)                                                                                                                          \\
     & \mathbb{P}(A\cup B) = \mathbb{P}(A)+\mathbb{P}(B) - \mathbb{P}(A\cap B)                                                                                    \\
     & \mathbb{P}(A_1 \cup \dots \cup A_n) \leq \mathbb{P}(A_1)+\dots \mathbb{P}(A_n)                                                                             \\
     & \mathbb{P}(B) \leq \mathbb{P}(A)                                                                    & \text{if } B\subseteq A                              \\
     & \mathbb{P}(A\backslash B) = \mathbb{P}(A)-\mathbb{P}(B)                                             & \text{if } B\subseteq A                              \\
\end{align*}

\subsection{Discrete Probability Spaces}
A discrete probability space has at most countably many different elements. As the outcomes exclude each other:
\noindent\begin{align*}
    \mathbb{P}(A) & = \mathbb{P}\left(\bigcup_{\substack{\omega_i \in A}}\{\omega_i\}\right)= \sum_{\substack{\omega_i \in A}} \mathbb{P}(\omega_i)
\end{align*}
\subsubsection{Laplace Model}
In the Laplace model, all possible outcomes have the same probability.
\noindent\begin{align*}
    \mathbb{P}(\omega_i) & = \frac{1}{|\Omega|}                                             \\
    \mathbb{P}(A)        & = \sum_{\omega_i \in A}\frac{1}{|\Omega|} = \frac{|A|}{|\Omega|}
\end{align*}

\section{Independence, Conditional Prob.\ and Bayes}
\subsection{Independence}
\ptitle{General}

\noindent\begin{align*}
    A \cap B = \emptyset                          & \Rightarrow \mathbb{P}(A\cap B) = 0                                 \\
    B \subseteq A                                 & \Rightarrow \mathbb{P}(A\cap B) = \mathbb{P}(B)                     \\
    0 \leq \mathbb{P}(A\cap B)                    & \leq \min\{\mathbb{P}(A), \mathbb{P}(B)\}                           \\
    \max\{0, 1-\mathbb{P}(A^C) -\mathbb{P}(B^C)\} & \leq \mathbb{P}(A\cap B) \leq \min\{ \mathbb{P}(A), \mathbb{P}(B)\}
\end{align*}

\ptitle{Independent}

$A$ and $B$ are independent if
\noindent\begin{equation*}
    \mathbb{P}(A\cap B) = \mathbb{P}(A)\mathbb{P}(B)
\end{equation*}
For several events $A_{1\dots n}$ and each choice $A_{i1\dots ik}$ with $k\leq n$
\noindent\begin{equation*}
    \mathbb{P}(A_{ij} \cap \dots \cap A_{ik}) = \mathbb{P}(A_{ij}) \cdots \mathbb{P}(A_{ik})
\end{equation*}
holds. This implies that each subset of events must be independent as well.

\subsection{Total Probability}
\noindent\begin{align*}
    \mathbb{P}(A)     & =\mathbb{P}(A\cap B)+\mathbb{P}(A\cap B^c)                                           \\
                      & =\mathbb{P}(A\mid B)\mathbb{P}(B)+\mathbb{P}(A\mid B^c)\mathbb{P}(B^c)               \\\\
    \mathbb{P}(A)     & =\sum_{i=1}^k\mathbb{P}(A\cap B_i) =\sum_{i=1}^k\mathbb{P}(A\mid B_i)\mathbb{P}(B_i) \\
    \text{if } \Omega & =B_1\cup\cdots\cup B_k,\quad\text{with }B_i\cap B_j=\emptyset\text{ for }i\neq j
\end{align*}
\subsection{Conditional Prob.\ and Bayes}
\noindent\begin{align*}
    \mathbb{P}(B\mid A) & =\frac{\mathbb{P}(A\cap B)}{\mathbb{P}(A)}=\frac{\mathbb{P}(A\mid B)\mathbb{P}(B)}{\mathbb{P}(A)} \overset{\text{indep.}}{=} \mathbb{P}(B)
\end{align*}
If $\Omega=B_1\cup\cdots\cup B_k\quad$ with $B_i\cap B_j=\emptyset$ for $i\neq j$, then by the \textit{law of total probability}:
\noindent\begin{align*}
    \mathbb{P}(B_i\mid A) & =\frac{\mathbb{P}(A\cap B_i)}{\mathbb{P}(A)}=\frac{\mathbb{P}(A\mid B_i)\mathbb{P}(B_i)}{\mathbb{P}(A)}                              \\
                          & =\frac{\mathbb{P}(A\mid B_i)\mathbb{P}(B_i)}{\sum_{j=1}^k \underbrace{\mathbb{P}(A\mid B_j)\mathbb{P}(B_j)}_{\mathbb{P}(A\cap B_j)}}
\end{align*}

\textbf{Remark:} The conditional probability $\mathbb{P}(\cdot|B)$ can be viewed as a new probability measure over $\Omega=B$. Thus, the usual probability rules stated in Subsubsection \ref{sssec:rules_from_axioms} hold.