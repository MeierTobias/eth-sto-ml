\section{Law of Large Numbers and the Central Limit Theorem}

\subsection{i.i.d Random Variables}

$X_1, \ldots , X_n$ are \textbf{independent identically distributed (i.i.d)} random variables.
\begin{gather*}
    S_n = X_1 + \cdots + X_n \\
    \bar{X}_n = \frac{S_n}{n}
\end{gather*}
Then $S_n$ and $ \bar{X}_n$ are also random variables.
\newpar{}
If $\mathbb{E}[X_1]=\mu$ and $\mathrm{Var}(X_1)=\sigma^2$ then
\begin{gather*}
    \mathbb{E}[S_n]=n\mu \\
    \mathbb{E}[\bar{X}_n]=\mu \\
    \mathrm{Var}(S_n)=n\sigma^2 \\
    \mathrm{Var}(\bar{X}_n)=\frac{\sigma^2}{n} \\
    \sigma_{S_n}=\sqrt{n}\sigma \\
    \sigma_{\bar{X}_n}=\frac{\sigma}{\sqrt{n}}
\end{gather*}

\subsection{Law of Large Numbers (LLN)}
$X_1, X_2, \ldots$ are i.i.d.\ with $\mathbb{E}[X_i]=\mu$ and $\mathrm{Var}(X_i)=\sigma^2$. Then
\begin{align*}
    \lim_{n\to\infty}\mathbb{P}\left(\left|\bar{X}_n-\mu\right|>\epsilon\right) & =0 \qquad \forall \epsilon > 0 \\
    \mathbb{P}\left(\lim_{n\to\infty}\bar{X}_n=\mu\right)                       & =1
\end{align*}

In other words, for a large $n$ the $\bar{X}_n$ becomes $\mu$.

\subsection[LLN Examples]{Distribution of $S_n$ and $\bar{X}_n$}
\noindent\begin{align*}
    X_i & \sim \mathrm{Ber}(p)            & \text{then} &  & S_n & \sim \mathrm{Bin}(n,p)            \\
    X_i & \sim \mathrm{Pois}(\lambda)     & \text{then} &  & S_n & \sim \mathrm{Pois}(n\lambda)      \\
    X_i & \sim \mathcal{N}(\mu, \sigma^2) & \text{then} &  & S_n & \sim \mathcal{N}(n\mu, n\sigma^2) \\
    X_i & \sim \mathrm{Exp}(\lambda)      & \text{then} &  & S_n & \sim \mathrm{Gamma}(n, \lambda)
\end{align*}

\subsection{Central Limit Theorem (CLT)}
$X_1, X_2, \ldots$ are i.i.d.\ with $\mathbb{E}[X_i]=\mu$ and $\mathrm{Var}(X_i)=\sigma^2$. Then
\begin{align*}
    S_n                                       & \approx \mathcal{N}(n\mu, n\sigma^2)         \\
    \bar{X}_n                                 & \approx \mathcal{N}(\mu, \frac{\sigma^2}{n}) \\
    \frac{1}{\sqrt{n}}\frac{S_n-n\mu}{\sigma} & \approx\mathcal{N}(1,0)
\end{align*}

\ptitle{Remarks}

\begin{itemize}
    \item Typically, the approximation is already good for $n \geq 30$.
    \item In words, the CLT states that for large $n$ the arithmetic mean of the experiments is normally distributed with the same mean as $X_i$ but with a smaller standard deviation than the individual $X_i$.
    \item Example: if one rolls infinitely many dices the arithmetic mean will be distributed as $\delta(x-3.5)$ i.e. with no variance.
\end{itemize}
\subsubsection{Normal Approximation of Binimial Distributions}
As $Bin(n,p)$ is the sum of $n$ independent $Ber(p)$ random variables one can apply the CLT for large $n$ i.e.:
\begin{align*}
    X                   & \approx\mathcal{N}(np,np(1-p))                                           \\
    \mathbb{P}(X\leq x) & \approx\Phi\left(\frac{x-np}{\sqrt{np(1-p)}}\right),\quad x\in\mathbb{R}
\end{align*}