\section{Statistical Inference}
Attempts to find the right model.
\subsection{Set-up}
\begin{itemize}
    \item Observed data $x_1,\ldots, x_n$ is viewed as realizations of i.i.d.\ random variables $X_1,\ldots, X_n$ with distribution $F_\theta$
    \item It is assumed that the model family $F_\theta$ is known and $\theta\in \mathbb{R}^r$ is a free parameter
    \item The distribution family is e.g. chosen based on experience, CLT or graphical comparisons.
    \item Goal: Infer $\theta$ from data
\end{itemize}

\subsection{Main Questions}
\begin{enumerate}
    \item What is the most plausible value of  $\theta\in \mathbb{R}^r$?\newline
          $\to$ \textit{point estimate (best guess)}
    \item Is a given value  $\theta_O\in \mathbb{R}^r$ compatible with the data?\newline
          $\to$ \textit{statistical test}
    \item What is a region of plausible parameter values  $\theta\in \mathbb{R}^r$?\newline
          $\to$ \textit{confidence intervals/ regions}
\end{enumerate}

\subsection{QQ Plot}
One way of comparing data with a chosen distribution family is by using a \textbf{QQ Plot},
where the quantiles of the data are compared with a standard normal distribution.


\renewcommand{\arraystretch}{1.3}
\setlength{\oldtabcolsep}{\tabcolsep}\setlength\tabcolsep{12pt}
\begin{tabularx}{\linewidth}{@{}llll@{}}
    $k$      & $\alpha_k = \frac{(k-0.5)}{n}$ & $\underbrace{q_{a_k} = x_k}_{\textsf{measured}}$ & $\underbrace{\Phi^{-1}(\alpha_k)}_{\textsf{theoretical}}$ \\
    \cmidrule{2-4}
    1        & 0.025                          & 24.4                                             & -1.96                                                     \\
    $\vdots$ & $\vdots$                       & $\vdots$                                         & $\vdots$                                                  \\
    $n$      & 0.975                          & 39.7                                             & 1.96                                                      \\
    \cmidrule{3-4}
             &                                & \multicolumn{2}{c}{QQ plot}
\end{tabularx}
\renewcommand{\arraystretch}{1}
\setlength\tabcolsep{\oldtabcolsep}

\textbf{Remarks}:
\begin{itemize}
    \item If the observations are realizations of the theoretical distribution, then the points on the QQ plot lie roughly on the diagonal.
    \item If the observations are realizations of an affine transformation $a+bX$ of the theoretical distribution, then the points on the QQ plot lie roughly on $y = a+bx$.
    \item The mean of the measured distribution is found at $\Phi^{-1}(\alpha_k)=0$.
    \item If the measured data are normally distributed, their standard deviation is the slope of the QQ plot.
    \item Depending on the distribution of the measured data, some values might not be taken (e.g. $~Unif$)!
    \item The tails of the measured distribution strongly influence the lower left and upper right values in the QQ plot.
\end{itemize}

\section{Point Estimates}
Point estimates answer the first main question of statistical inference.
A \textit{point estimation} can be either
\begin{itemize}
    \item A function $\hat{\theta}: \mathbb{R}^n\to \mathbb{R}^r$
    \item A best guess $\hat{\theta} = \hat{\theta}(x_1,\ldots, x_n)$
    \item A random variable/ vector $\hat{\theta} = \hat{\theta}(X_1,\ldots, X_n)$
\end{itemize}

\subsection{Method of Moments (MoM)}
The \textbf{moments} of a random variable $X$ are
\noindent\begin{equation*}
    \mu_k=\mathbb{E}[X^k],k=1,2,\ldots
\end{equation*}
These moments $\mu_k$ can be estimated with \textit{empirical moments} $m_k$
\noindent\begin{equation*}
    m_k=\frac{1}{n}\sum_{i=1}^n {x_i}^k
\end{equation*}

\begin{examplesection}[Examples of MoM]
    $X_i \sim$ Pois($\lambda$)
    \noindent\begin{align*}
        \lambda       & = \mathbb{E}[X_i] = \mu_i            \\
        \hat{\lambda} & = m_1 =\frac{1}{n}\sum_{i=1}^n {x_i}
    \end{align*}
    or
    \noindent\begin{align*}
        \lambda       & = \mathrm{Var}(X_i)=\mu_2-\mu_1^2                                                     \\
        \hat{\lambda} & =m_2-m_1^2=\frac{1}{n}\sum_{i=1}^n x_i^2-{\left(\frac{1}{n}\sum_{i=1}^n x_i\right)}^2
    \end{align*}

    $X_i \sim \mathcal{N}(\mu,\sigma^2)$
    \noindent\begin{align*}
        \mu            & = \mu_1, \quad \sigma^2 = \mu_2-{\mu_1}^2                                             \\
        \hat{\mu}      & =m_1=\frac{1}{n}\sum_{i=1}^n x_i                                                      \\
        \hat{\sigma}^2 & =m_2-m_1^2=\frac{1}{n}\sum_{i=1}^n x_i^2-{\left(\frac{1}{n}\sum_{i=1}^n x_i\right)}^2
    \end{align*}
\end{examplesection}

\subsection{Maximum Likelihood Estimation (MLE)}
\begin{enumerate}
    \item choose distribution
    \item Define \textit{Likelihood} as a function of parameter $\lambda$
          \noindent\begin{equation*}
              L(\lambda) = \mathbb{P}(x_1,\ldots ,x_n|\lambda)
          \end{equation*}
    \item Find the \textit{maximum likelihood estimator}
          \noindent\begin{equation*}
              \hat{\lambda} = \mathrm{argmax}_\lambda L(\lambda)
          \end{equation*}
    \item (optional) define \textit{log-likelihood function}
          \noindent\begin{align*}
              l(\lambda)    & = \ln(L(\lambda))                    \\
              \hat{\lambda} & = \mathrm{argmax}_\lambda l(\lambda)
          \end{align*}
\end{enumerate}

\begin{examplesection}[Examples of MLE]
    $X_i \sim$ Pois($\lambda$)
    \noindent\begin{align*}
        L(\lambda)          & =\prod_{i=1}^n e^{-\lambda}\frac{\lambda^{x_i}}{x_i!}                                                                               \\
        l(\lambda)          & =\sum_{i=1}^n\log\left(e^{-\lambda}\frac{\lambda^{x_i}}{x_i!}\right)=\sum_{i=1}^n\left\{x_i\log(\lambda)-\log(x_i!)-\lambda\right\} \\
        l^{\prime}(\lambda) & = 0 \Leftrightarrow \sum_{i=1}^n\frac{x_i}\lambda-n = 0 \Rightarrow \hat{\lambda} =\frac1n\sum_{i=1}^n x_i
    \end{align*}
\end{examplesection}

\subsubsection{Continuous Distribution with a Density}
For continuous distributions with a density, the MLE can be found with
\noindent\begin{equation*}
    \hat{\theta}=\operatorname{argmax}_\theta\sum_{i=1}^n\log f_\theta(x_i)
\end{equation*}
If $\theta$ is $r$-dimensional, the all partial derivatives can be set to 0:
\noindent\begin{equation*}
    \frac{\partial l(\theta)}{\partial\theta_1}=\cdots=\frac{\partial l(\theta)}{\partial\theta_r}=0\quad\leadsto\text{this gives }r\text{ equations}
\end{equation*}

\begin{examplesection}[Examples]
    $X_n\sim\mathcal{N}(\mu,\sigma^2=v)$
    \noindent\begin{align*}
        -l(\mu,v)                                & =\frac12\sum_{i=1}^n\left\{\frac{{(x_i-\mu)}^2}v+\log(2\pi)+\log(v)\right\}         \\
        0=-\frac{\partial l(\mu,v)}{\partial\mu} & =\sum_{i=1}^n\frac{\mu-x_i}v\quad\Rightarrow\quad \hat{\mu}=\frac1n\sum_{i=1}^n x_i \\
        0=-\frac{\partial l(\mu,v)}{\partial v}  & =\frac12\sum_{i=1}^n\left\{-\frac{{(x_i-\mu)}^2}{v^2}+\frac1v\right\}               \\
                                                 & \Rightarrow\quad\hat{v}=\frac1n\sum_{i=1}^n{(x_i-\hat{\mu})}^2
    \end{align*}
    $X_n\sim \mathrm{Unif}(0,\theta)$
    \noindent\begin{align*}
        f_\theta(x)        & =\begin{cases}
                                  1/\theta & \quad\mathrm{if~}0\leq x\leq\theta \\
                                  0        & \quad\mathrm{else}
                              \end{cases}                                                  \\
        L(\theta)          & =\prod_{i=1}^n f_\theta(x_i)=\begin{cases}
                                                              \prod_{i=1}^n1/\theta=1/\theta^n & \mathrm{if~}0\leq x_i\leq\theta \\
                                                              0                                & \mathrm{else}
                                                          \end{cases} \\
        \theta(x_1,\ldots) & =\max_i x_i \text{ all } x_i \text{ have to lie within } [0,\theta]
    \end{align*}
\end{examplesection}

\subsection{General Estimators}
For i.i.d.\ random variables with \textit{unknown distributions} the \textit{general estimators}
\noindent\begin{align*}
    \hat{\mu}_{X}    & =\bar{X}_{n}=\frac1n\sum_{i=1}^{n}X_{i}             \\
    \hat{\sigma}_X^2 & =s_n^2=\frac1{n-1}\sum_{i=1}^n{(X_i-\hat{\mu}_X)}^2
\end{align*}
can be used. These estimators have the following properties:
\noindent\begin{align*}
    \mathbb{E}[\hat{\mu}_x]                                       & =\mathbb{E}[\bar{X}_n]=\frac1n\sum_{i=1}^n\mathbb{E}[X_i]=\mu_X                                    \\
    \mathrm{Var}(\hat{\mu}_X)                                     & =\mathrm{Var}(\bar{X}_n)=\frac{n\mathrm{Var}(X_1)}{n^2}=\frac{\sigma_X^2}n                         \\
    \underbrace{\sigma_{\hat{\mu}_{X}}}_{\textsf{of mean estim.}} & =\sqrt{\mathrm{Var}(\hat{\mu}_{X})}=\underbrace{\frac{\sigma_{X}}{\sqrt{n}}}_{\textsf{of dataset}} \\
    \mathbb{E}[\hat{\sigma}_X^2]                                  & =\frac1{n-1}\mathbb{E}\left[\sum_{i=1}^n{\left(X_i-\bar{X}_n\right)}^2\right] = \sigma^2_X
\end{align*}

\textbf{Remark} $\hat{\mu}_X, \hat{\sigma}_X^2$ are unbiased
