\section{Confidence Intervals}
% Confidence intervals answer the third main question of statistical inference.
A $(1-\alpha)\times 100\:\%$-confidence interval for a parameter $\theta$ is a random interval $I$ satisfying $\mathbb{P}(\theta\in I)=1-\alpha$.

The $(1-\alpha)\times 100\:\%$-confidence interval for a normally distributed random variable is given by
\begin{equation*}
    I=\bar{x}_n \pm \frac{\sigma}{\sqrt{n}}z_{1-\alpha/2}
\end{equation*}
similar for the $t$-distribution
\begin{equation*}
    I=\bar{x}_n \pm \frac{s_n}{\sqrt{n}}t_{\nu,1-\alpha/2}
\end{equation*}

\newpar{}
\ptitle{Duality between Confidence Intervals and Tests}

\begin{itemize}
    \item Use tests as confidence intervals: For a two-sided test with significance level $\alpha$ one has that
    \begin{equation*}
        I=\{\theta_0:\text{ the null hypothesis }H_0\colon\theta=\theta_0\text{ is not rejected}\}
    \end{equation*}
    is a $(1-\alpha)\times 100\:\%$-confidence interval (all $\theta_0$ are tested with $\alpha$ and those are collected for wich $H_0$ is not rejected).
    \item Use confidence intervals as tests: If a confidence interval contains $\theta_0$ then $H_0$ is accepted.
\end{itemize}

% \newpar{}
% \ptitle{Significance vs. Relevance}

% A significant effect does not necessary mean the result is relevant. This depends on the application an cannot be answered by statistics.
% \newpar{}
% Example:

% If we test the null hypothesis $H_0:\mu=400$, but the true parameter is $\mu=401$, then with a sufficiently large sample, we will obtain a statistically significant result but it may be irrelevant for the application.

% \begin{center}
%     \includegraphics[width=\linewidth]{significance_relevance.png}
% \end{center}

