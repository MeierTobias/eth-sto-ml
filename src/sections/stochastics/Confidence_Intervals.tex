\section{Confidence Intervals}
Confidence intervals answer the third main question of statistical inference.
A $(1-\alpha)\times 100\%$-confidence interval for a parameter $\theta$ is a random interval $I$ satisfying $\mathbb{P}(\theta\in I)=1-\alpha$.

The $(1-\alpha)\times 100\%$-confidence interval for a normal distributed random variable is given by
\begin{equation*}
    I=\bar{x}_n \pm \frac{\sigma}{\sqrt{n}}z_{1-\alpha/2}
\end{equation*}
similar for the $t$-distribution
\begin{equation*}
    I=\bar{x}_n \pm \frac{\sigma}{\sqrt{n}}t_{\nu,1-\alpha/2}
\end{equation*}

\newpar{}
\ptitle{Duality between Confidence Intervals and Tests}

If a confidence interval contains $\theta_0$ then the $H_0$ is accepted.

\newpar{}
\ptitle{Significance vs. Relevance}

A significant effect does not necessary mean the result is relevant. This depends on the application an cannot be answered by statistics.
\newpar{}
Example:

If we test the null hypothesis $H_0:\mu=400$, but the true parameter is $\mu=401$, then with a sufficiently large sample, we will obtain a statistically significant result but it may be irrelevant for the application.