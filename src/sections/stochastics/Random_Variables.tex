\section{Random Variables}
When using random variables, the event $A$ is a subset of $\mathbb{R}$, and its probability given by
\noindent\begin{align*}
    X : \Omega         & \rightarrow \mathbb{R}                                                                       \\
    \mathbb{P}(X\in A) & =\mathbb{P}(\underbrace{\{\omega\in\Omega:X(\omega)\in A\}}_{=X^{-1}(A)\text{ := preimage}}) % ChkTex 26
\end{align*}
Thus, events can be defined using intervals:
\noindent\begin{align*}
    \mathbb{P}(a\leq X\leq b) & =\mathbb{P}(X\in[a,b])
\end{align*}

\subsection{Discrete Random Variables}
\noindent\begin{equation*}
    X:\Omega\rightarrow W=\{x_1,x_2,x_3,\ldots\}\subseteq\mathbb{R}
\end{equation*}
A random variable $X$ is said to be \textit{discrete} if its \textit{range} $W$ is \textbf{at most countable}.
\newpar{}
\ptitle{PMF}

The \textit{probability mass function (pmf)} of $X$ is given by
\begin{equation*}
    p(x_{k}) =\mathbb{P}(X=x_{k})
\end{equation*}
For the pmf one then has the properties
\noindent\begin{align*}
    \mathbb{P}(X\in A)  & =\sum_{k:x_k\in A}p(x_k)                        \\
    \sum_{k\geq1}p(x_k) & =1                       & \text{Normalization}
\end{align*}

\ptitle{Distribution}

The \textit{distribution} of $X$ describes probabilities for subsets (e.g.\ intervals) on the real line. It is the probability measure $\mathbb{Q}$ on $\mathbb{R}$ given by
\noindent\begin{equation*}
    \mathbb{Q}(A)=\mathbb{P}(X\in A)=\sum_{k:x_k\in A} \underbrace{p(x_k)}_{\text{pmf}}\quad\mathrm{for~}A\subseteq\mathbb{R}
\end{equation*}

\ptitle{CDF}

The \textit{cumulative distribution function (cdf)} of $X$ is the function
\noindent\begin{align*}
    F:\mathbb{R}            & \rightarrow [0,1]                             \\
    F(x)=\mathbb P(X\leq x) & =\sum_{k:x_k\leq x}p(x_k),\quad x\in\mathbb R
\end{align*}

\subsubsection{Properties of CDFs}
\begin{itemize}
    \item F is non-decreasing (weakly increasing)
    \item $\lim_{x\rightarrow -\infty}F(x)=0$ and $\lim_{x\rightarrow \infty}F(x)=1$
    \item F is right-continuous.\ i.e. $\lim_{y\downarrow x} F(y) =F(x)$ (i.e.\ $F(y)$ converges towards $F(x)$ when $y>x$ converges towards $x$)
\end{itemize}

\subsubsection{General Rules for Discrete Random Variables}
\noindent\begin{align*}
    \mathbb{P}(X>x)       & =1-\mathbb{P}(X\leq x)=1-F(x)                                   \\
    \mathbb{P}(X\geq x)   & =\mathbb{P}(X>x)+\mathbb{P}(X=x)=1-F(x)+p(x)                    \\
    \mathbb{P}(a<X\leq b) & =\mathbb{P}(X\leq b)-\mathbb{P}(X\leq a)=F(b)-F(a)              \\
    p(x_k)                & \overset{\text{adjacent in } W}{=}\mathbb{P}(x_{k-1}<X\leq x_k) \\
                          & =F(x_k)-F(x_{k-1})
\end{align*}

\subsection{Independence of Random Variables}
Two random variables $X,Y_\Omega \rightarrow\mathbb{R}$ are \textbf{independent} if
\noindent\begin{align*}
    \mathbb{P}(X\in A,Y\in B)            & =                                                                        \\
    % is the overset{!}{=} below really necessary?
    %\mathbb{P}(\{X\in A\}\cap\{Y\in B\}) & \overset{!}{=}\mathbb{P}(X\in A)\mathbb{P}(Y\in B)\quad\forall A,B\subseteq\mathbb{R}
    \mathbb{P}(\{X\in A\}\cap\{Y\in B\}) & =\mathbb{P}(X\in A)\mathbb{P}(Y\in B)\quad\forall A,B\subseteq\mathbb{R}
\end{align*}

Several random variables $X_a, \dots X_n :\Omega\rightarrow\mathbb{R}$ are \textbf{independent} if
\noindent\begin{align*}
    \mathbb{P}(X_1\in A_1,\ldots,X_n\in A_n) = & \mathbb{P}(X_1\in A_1)\cdots\mathbb{P}(X_n\in A_n) \\
                                               & \forall A_1,\ldots,A_n\subseteq\mathbb{R}
\end{align*}
% TBD: Again, all subsets of events must be independent, right?

\subsection{Statistical Moments}
Statistical moments are important summary figures to describe distributions but do not determine the whole distribution. For unique representation $p$ and $F$ are required.
\subsubsection{Expectation}
The \textbf{expectation} of a discrete random variable $X$
\noindent\begin{equation*}
    \mu_{X}=\mathbb{E}[X]=\sum_{k\geq1}x_{k}p(x_{k})\in\mathbb{R}
\end{equation*}
can be interpreted as the
\begin{itemize}
    \item mean location of the distribution
    \item average under a large number of repetitions
    \item center of mass of the pmf.
\end{itemize}

\newpar{}
% TBD: I guess, we can again use the 2nd formula "E[g^2] - E[g]^2" too?
If $Y=g(X)$ is a \textbf{transformation} of the random variable $X$, its \textbf{expectation value} is given by
\noindent\begin{align*}
    \mathbb{E}[Y]    & =\mathbb{E}[g(X)]=\sum_{k\geq1}g(x_{k})p(x_{k})                    \\
    \mathbb{E}[g(X)] & \neq g(\mathbb{E}[X])                           & \text{generally}
\end{align*}

\subsubsection{Properties of the Expectation}
These properties also hold for non-discrete random variables if $\mathbb{E}[X]$ and $\mathbb{E}[Y]$ are well-defined.
\noindent\begin{align*}
    \textbf{Constants:}                   &  &  & a\in\mathbb{R},\mathbb{E}[a]=a\cdot p(a)=a                                          \\
    \textbf{Linearity:}                   &  &  & a\mathbb{E}[X]=\mathbb{E}[aX],\;\;a\in \mathbb{R}                                   \\
                                          &  &  & \mathbb{E}[X]+\mathbb{E}[Y]=\mathbb{E}[Z],\;\; Z=X+Y                                \\
    \textbf{Monotonicity:}                &  &  & \mathbb{E}[X]=\sum_{k\geq1}x_k\mathbb{P}(X=x_k)\geq0, \; W_x \subseteq \mathbb{R}_+ \\
    X(\omega)\geq Y(\omega)\forall\omega: &  &  & \operatorname{E}[X]-\operatorname{E}[Y]=\operatorname{E}[X-Y]\geq0                  \\\\
    \text{Assuming}                       &  &  & \text{independence of } X,Y:                                                        \\
    \textbf{Product:}                     &  &  & \mathbb{E}[X]\mathbb{E}[Y] = \mathbb{E}[XY] = \mathbb{E}[Z],                        \\
                                          &  &  & Z=XY, \;W_Z =\{x_1y_1,x_1y_2,\ldots,x_2y_1,\ldots\}                                 \\
                                          &  &  & \text{which can be extended to }X_1,\ldots,X_n
\end{align*}

\subsubsection{Variance}
The \textbf{variance} of a discrete random variable
\noindent\begin{align*}
    \mathrm{Var}(X) & = \mathbb{E}[{(X-\mathbb{E}[X])}^2] = \mathbb{E}[X^2]-{\mathbb{E}[X]}^2 \\
                    & = \sum_{k\geq1}{(x_k-\mu_X)}^2p(x_k)\geq0
\end{align*}
can be interpreted as the
\begin{itemize}
    \item mean quadratic deviation from the mean
    \item measure of dispersion
\end{itemize}

\newpar{}
If $Y=g(X)$ is a \textbf{transformation} of the random variable $X$, its \textbf{variance} is given by
\noindent\begin{align*}
    \mathrm{Var}(g(X)) & =\mathbb{E}[{(g(X)-\mathbb{E}[g(X)])}^{2}]   = \mathbb{E}[{g(X)}^2]-{\mathbb{E}[g(X)]}^2 \\
                       & =\sum_{k\geq1}{(g(x_{k})-\mathbb{E}[g(X)])}^{2}p(x_{k})                                  \\
    \mathrm{Var}(g(X)) & \neq g(\mathrm{Var}(X)) \qquad \text{generally}
\end{align*}

\subsubsection{Standard Deviation}
The \textbf{standard deviation} is the square root of the \textbf{variance}
\noindent\begin{equation*}
    \sigma_{X}=\sqrt{\mathrm{Var}(X)}
\end{equation*}
and has the same unit as $X$.

\subsubsection{Properties of Variance and Standard Deviation}
\ptitle{Variance}

The variance is \textbf{not linear}.

(Assuming all expectations are well defined)
\noindent\begin{align*}
    \mathrm{Var}(X)\geq0 & \Rightarrow\mathbb{E}[X^{2}]\geq{\mathbb{E}[X]}^{2}        \\
    \mathrm{Var}(a)      & =\mathbb{E}[{(a-\mathbb{E}[a])}^2]=\mathbb{E}[{(a-a)}^2]=0 \\
    \mathrm{Var}(a+bX)   & = b^2\mathrm{Var}(X)\text{ (invariant under translation)}
\end{align*}
\ptitle{Variance of Sums}

\noindent\begin{align*}
    \mathrm{Var}(a_1X_1 & +\cdots+a_n X_n)  =\sum_{i=1}^{n}a_{i}^{2}\mathrm{Var}(X_{i})+\ldots                  \\
                        & + \underbrace{2\sum_{i<j}a_{i}a_{j}\mathrm{Cov}(X_{i},X_{j})}_{0\text{ if indp.}}     \\
    \mathrm{Var}(X+Y)   & =\mathrm{Var}(X)+\mathrm{Var}(Y)+ \underbrace{2\mathrm{Cov}(X,Y)}_{0\text{ if indp.}} \\
    \mathrm{Var}(X-Y)   & =\mathrm{Var}(X)+\mathrm{Var}(Y)-\underbrace{2\mathrm{Cov}(X,Y)}_{0\text{ if indp.}}
\end{align*}
\textbf{Caution}: $\mathrm{Var}(X\pm Y) = \mathrm{Var}(X)+\mathrm{Var}(Y)$ does not necessarily mean independence because $\mathrm{Cov}(X,Y)=0$ can also hold for perfect nonlinear dependence: \textbf{independence is much stronger} than uncorrelatedness.

\newpar{}
\ptitle{Standard Deviation}
\noindent\begin{equation*}
    \sigma_{a+bX}=\sqrt{\mathrm{Var}(a+bX)}=\sqrt{b^2\mathrm{Var}(X)}=|b|\sigma_X
\end{equation*}

\subsubsection{Covariance and Correlation}
The \textbf{covariance} between two (discrete) random variables $X,Y$
\noindent\begin{align*}
    \operatorname{Cov}(X,Y) & =\mathbb{E}[(X-\mathbb{E}[X])(Y-\mathbb{E}[Y])] \\
                            & = \mathbb{E}[XY]-\mathbb{E}[X]\mathbb{E}[Y]
\end{align*}
\begin{itemize}
    \item is (only) a measure of \textbf{linear dependence} between $X$, $Y$
    \item can not tell wether $X$ causes $Y$, $Y$ causes $X$, or both are caused by a third variable $Z$
    \item if $\operatorname{Cov}(X,Y)$ and a certain $x$ are given we can conclude on where we expect $y$ to be with respect to $\mathbb{E}[Y]$ (and vice versa)
\end{itemize}

\newpar{}
The \textbf{correlation} between two (discrete) random variables $X,Y$
\noindent\begin{equation*}
    \mathrm{Corr}(X,Y)=\rho_{XY}=\frac{\mathrm{Cov}(X,Y)}{\sigma_{X}\sigma_{Y}}
\end{equation*}
\begin{itemize}
    \item is a \textbf{normalized} version of the covariance:\newline$-1 \leq \rho_{XY} \leq 1$
    \item \textbf{Perfect positive correlation}\newline $\rho_{XY}=1$ if and only if $Y=a+bX\mathrm{~for~}b>0$
    \item \textbf{Perfect negative correlation}\newline $\rho_{XY}=-1$ if and only if $Y=a+bX\mathrm{~for~}b<0$
    \item if e.g.\ $Y=\exp(X)$, one has perfect dependence but $\mathrm{Corr}(X,Y)<1$
    \item $\mu_X, \sigma_X$ give information about the marginal distribution of $X$
\end{itemize}

Caution:
\begin{itemize}
    \item $\rho_{XY}<1$ (or even $\rho_{XY}=0$) only means that there is no perfect linear dependence but there could still be a perfect nonlinear dependence!
\end{itemize}

\subsubsection{Properties of the Covariance and Correlation}
(Assuming all expectations are well defined)
\noindent\begin{align*}
    % \mathrm{Cov}(X,Y)        & =\mathbb{E}[(X-\mathbb{E}[X])(Y-\mathbb{E}[Y])]             \\
    %                          & = \mathbb{E}[XY]-\mathbb{E}[X]\mathbb{E}[Y]                 \\
    \mathrm{Cov}(X,X)        & = \mathrm{Var}(X)                                           \\
    \operatorname{Cov}(X,Y)  & = 0 \Leftarrow X,Y\text{ idp.}                              \\
    \mathrm{Cov}(a+bX,c+dY)  & =bd\operatorname{Cov}(X,Y),\quad b,d\in\mathbb{R}           \\
    \mathrm{Corr}(a+bX,c+dY) & ={\frac{bd}{|bd|}}\mathrm{Corr}(X,Y),\quad b,d\in\mathbb{R}
\end{align*}

\subsection{Discrete Distributions}

\subsubsection{Bernoulli Distribution}
$X \sim \mathrm{Ber}(p)$

\renewcommand{\arraystretch}{1.3}
\setlength{\oldtabcolsep}{\tabcolsep}\setlength\tabcolsep{3pt}

\begin{tabularx}{\linewidth}{@{}p{0.5\linewidth}p{0.49\linewidth}@{}}
    $W=\{0,1\}$                                     &
    \multirow{4}{*}{
        \begin{tikzpicture}
    % \pgfplotsset{ticks = none}
    \tiny
    \begin{axis}[
            % xlabel={$x$},
            ylabel={Probability},
            legend style={at={(1,1)},anchor=north east},
            legend style={font=\tiny},
            ymin  = 0,
            yticklabel=\empty,
            ytick = \empty,
            xtick={0,1},
            height = 3cm,
            width = 5cm,
            grid style=dashed,
            smooth,
        ]
        \addplot [
            domain=0:1,
            samples=2,
            color=red,
            ycomb,
            line width = 2pt,
        ]
        {((1-0.2)^(x-1))*0.2};
    \end{axis}
\end{tikzpicture}
    }                                                 \\
    $\mathbb{P}(X=0)=1-p,\newline\mathbb{P}(X=1)=p$ & \\
    $\mathbb{E}[X] = p$                             & \\
    $\mathrm{Var}(X) = p(1-p)$                      &
\end{tabularx}

\renewcommand{\arraystretch}{1}
\setlength\tabcolsep{\oldtabcolsep}



\subsubsection{Binomial Distribution}
$X \sim \mathrm{Bin}(n,p)$\\
$X$ = number of successes in $n$ \textbf{independent} Bernoulli experiments.

\renewcommand{\arraystretch}{1.3}
\setlength{\oldtabcolsep}{\tabcolsep}\setlength\tabcolsep{0pt}

\begin{tabularx}{\linewidth}{@{}p{0.5\linewidth}p{0.49\linewidth}@{}}
    $W=\{0,1,\ldots,n\}$                                                                                             &
    \multirow{4}{*}{
        \begin{minipage}{\linewidth}
            \begin{tikzpicture}
    % \pgfplotsset{ticks = none}
    \tiny
    \begin{axis}[
            xlabel={$x$},
            ylabel={Probability},
            legend style={at={(1,1)},anchor=south east},
            legend style={font=\tiny},
            ymin  = 0,
            xtick={0,5,15},
            xticklabels={0,$n_1p_1$,$n_2p_2$},
            ytick = \empty,
            yticklabel=\empty,
            height = 3cm,
            width = 5cm,
            grid style=dashed,
            bar width=1pt,
        ]
        \addplot [
            domain=0:30,
            samples=30,
            color=red,
            ybar,
            draw opacity=1,
            line width = 1pt,
        ]
        {factorial(150)/(factorial(x)*factorial(150-x))*(0.1^x)*((0.9)^(150-x))};
        \addlegendentry{$n_1=150, p_1=0.1$}

        \addplot [
            domain=0:30,
            samples=30,
            color=blue,
            ybar,
            draw opacity=0.5,
            line width = 1pt,
        ]
        {factorial(50)/(factorial(x)*factorial(50-x))*0.1^x*(0.9)^(50-x)};
        \addlegendentry{$n_2=50,p_2=0.1$}
    \end{axis}
\end{tikzpicture}
        \end{minipage}
    }                                                                                                                  \\
    $\mathbb{P}(X=k)={n\choose k}p^k{(1-p)}^{n-k}\newline k\in {0,1,\ldots, n},\; {n\choose k}=\frac{n!}{k!(n-k)!} $ & \\
    $\mathbb{E}[X] = np$                                                                                             & \\
    $\mathrm{Var}(X) = np(1-p)$                                                                                      &
\end{tabularx}

\renewcommand{\arraystretch}{1}
\setlength\tabcolsep{\oldtabcolsep}

\subsubsection{Geometric Distribution}
$X \sim \mathrm{Geom}(p)$\\
$X$ = number of \textbf{independent} Bernoulli trials until first success.

\renewcommand{\arraystretch}{1.3}
\setlength{\oldtabcolsep}{\tabcolsep}\setlength\tabcolsep{3pt}

\begin{tabularx}{\linewidth}{@{}p{0.5\linewidth}p{0.49\linewidth}@{}}
    $W=\{1,2,\ldots\}$                                      &
    \multirow{4}{*}{
        \begin{tikzpicture}
    % \pgfplotsset{ticks = none}
    \tiny
    \begin{axis}[
            xlabel={$x$},
            ylabel={Probability},
            legend style={at={(1,1)},anchor=north east},
            legend style={font=\tiny},
            ymin  = 0,
            ytick = \empty,
            yticklabel=\empty,
            height = 3cm,
            width = 5cm,
            grid style=dashed,
            bar width=1pt,
        ]
        \addplot [
            domain=1:15,
            samples=15,
            color=red,
            ybar,
            draw opacity=1,
            line width = 2pt,
        ]
        {((1-0.2)^(x-1))*0.2};
        \addlegendentry{$p_1=0.2$}

        \addplot [
            domain=1:15,
            samples=15,
            color=blue,
            ybar,
            draw opacity=0.5,
            line width = 2pt,
        ]
        {((1-0.5)^(x-1))*0.5};
        \addlegendentry{$p_2=0.5$}
    \end{axis}
\end{tikzpicture}
    }                                                         \\
    $\mathbb{P}(X=k)={(1-p)}^{k-1}p\newline k\in\mathbb{N}$ & \\
    $\mathbb{E}[X] = \frac{1}{p}$                           & \\
    $\mathrm{Var}(X) = \frac{1-p}{p^2}$
\end{tabularx}

\renewcommand{\arraystretch}{1}
\setlength\tabcolsep{\oldtabcolsep}
\begin{itemize}
    \item The geometric distribution is \textbf{memoryless}.
    \item For $Y$=no success in $k$ experiments: $P(Y=k)={(1-p)}^k$.
\end{itemize}

\subsubsection{Poisson Distribution}
$X \sim \mathrm{Pois}(\lambda)$

\renewcommand{\arraystretch}{1.3}
\setlength{\oldtabcolsep}{\tabcolsep}\setlength\tabcolsep{3pt}
\begin{tabularx}{\linewidth}{@{}p{0.5\linewidth}p{0.49\linewidth}@{}}
    $W=\{0,1,2,\ldots\}$                                                      &
    \multirow{4}{*}{
        \begin{tikzpicture}
    \tiny
    \begin{axis}[
            xlabel={$x$},
            ylabel={Probability},
            legend style={at={(1,1)},anchor=north east},
            legend style={font=\tiny},
            xtick={0,3,9},
            xticklabels={0,$\lambda_1$,$\lambda_2$},
            ymin  = 0,
            ytick = \empty,
            yticklabel=\empty,
            height = 3cm,
            width = 5cm,
            grid style=dashed,
            bar width=1pt,
        ]
        \addplot [
            domain=0:25,
            samples=25,
            color=red,
            ybar,
            draw opacity=1,
            line width = 2pt,
        ]
        {exp(-3)*(3^x)/(factorial(x))};
        \addlegendentry{$\lambda_1 = 3$}

        \addplot [
            domain=0:25,
            samples=25,
            color=blue,
            ybar,
            draw opacity=0.5,
            line width = 2pt,
        ]
        {exp(-8)*(8^x)/(factorial(x))};
        \addlegendentry{$\lambda_2 = 8$}
    \end{axis}
\end{tikzpicture}
    }                                                                           \\
    $\mathbb{P}(X=k)=e^{-\lambda}\frac{\lambda^k}{k!},\quad k\in\mathbb{N}_0$ & \\
    $\mathbb{E}[X] = \lambda$                                                 & \\
    $\mathbb{E}[X(X-1)] = \lambda^2$                                          & \\
    $\mathrm{Var}(X) = \lambda$                                               &
\end{tabularx}
\renewcommand{\arraystretch}{1}
\setlength\tabcolsep{\oldtabcolsep}
\textbf{Remark:} For large $n$ and small $p$, $\mathrm{Bin}(n,p)\simeq \mathrm{Pois}(\lambda)$, $\lambda=np$.

For independent $X \sim \mathrm{Pois}(\lambda_1)$ and $Y \sim \mathrm{Pois}(\lambda_2)$ one has $X+Y \sim \mathrm{Pois}(\lambda_1 +\lambda_2)$

\subsection{Continuous Random Variables}
A random variable $X$ is continuous, if
\noindent\begin{equation*}
    F(x)=\mathbb{P}(X\leq x)
\end{equation*}
is continuous in $x$.


\subsubsection{Densities}
An integrable function $f:\mathbb{R}\rightarrow\mathbb{R}_+$ is a \textit{probability density} of $X$ if
\noindent\begin{equation*}
    \underbrace{F(x)}_{cdf} := \int_{-\infty}^{x} \underbrace{f(u)}_{pdf}\; du,\;\; \forall x\in \mathbb{R}
\end{equation*}
\begin{itemize}
    \item If $f$ is not continuous, $F$ is not differentiable everywhere
    \item If $F$ is differentiable, then $f = F'$
    \item Density $\Rightarrow$ continuous (opposite not given)
    \item In contrast to the discrete case, $f(x)>1$ is possible
    \item $f$ must be \textbf{normalized} and satisfy\newline $f(x)\geq 0 \; \forall x\in\mathbb{R}$
\end{itemize}
\noindent\begin{align*}
    \mathbb{P}(a\leq X\leq b)=\mathbb{P}(a<X\leq b)=F(b)-F(a)=\int_a^b f(u)du
\end{align*}
\subsubsection{Expectation and Variance}
\noindent\begin{align*}
    \mu_X=\mathbb{E}[X]                                 & =\int_{-\infty}^{\infty}xf(x)dx                               \\
    \mu_Y=\mathbb{E}[Y]                                 & =\int_{-\infty}^{\infty}g(x)f(x)dx,\;Y=g(X)                   \\
    \mathrm{Var}(X)=\mathbb{E}[{(X-\mu_{X})}^{2}]       & =\int_{-\infty}^{\infty}{(x-\mu_{X})}^{2}f(x)dx               \\
    \mathrm{Var}(Y)=\operatorname{E}[{(Y-\mu_{Y})}^{2}] & =\int_{-\infty}^{\infty}{\left(g(x)-\mu_{Y}\right)}^{2}f(x)dx
\end{align*}
\ptitle{Properties of Expectation and Variance}\\
The same properties as for discrete random variables hold, for example:
\noindent\begin{align*}
    \mathbb{E}[a+bX+cY]               & =a+b\operatorname{E}[X]+c\operatorname{E}[Y]                                             \\
    \mathbb{E}[X]\geq0\quad\text{if } & X\geq0\quad\Leftrightarrow f(x) = 0 \text{ for } x<0                                     \\
    \mathrm{Var}(X)                   & =\mathbb{E}[X^{2}]-{\mathbb{E}[X]}^{2}=\int_{-\infty}^{\infty}x^{2}f(x)dx-\mu_{X}^{2}    \\
    \mathrm{Var}(Y)                   & =\mathbb{E}[Y^{2}]-{\mathbb{E}[Y]}^{2}=\int_{-\infty}^{\infty}g^{2}(x)f(x)dx-\mu_{Y}^{2}
\end{align*}

\subsubsection{Quantiles}
\begin{itemize}
    \item $q_\alpha$ is an $\alpha$-quantile if $\mathbb{P}(X\leq q_\alpha)=\alpha,\text{ i.e. }F(q_\alpha)=\alpha$
    \item a $0.5$-quantile is called median\newline ($f$ symmetric $\leftrightarrow E[X]=$ median)
    \item the $q_{0.25}$ and $q_{0.75}$ are called lower and upper quartile
\end{itemize}
\ptitle{Discrete Random Variables}

For a discrete random variable $X$, $q_\alpha$ is an $\alpha$-quantile if
\begin{equation*}
    \mathbb{P}(X<q_\alpha)\le\alpha\le\mathbb{P}(X\le q_\alpha)
\end{equation*}
i.e. $q_\alpha$ is an interval where $F$ is equal to (value taken) or an $x$-value where $F$ jumps over (value not taken) the level $\alpha$.\\
% TBD: verify statements below
If the value of $\alpha$ is not taken this corresponds to
\begin{equation*}
    F(q_{\alpha-})< \alpha < F(q_{\alpha})
\end{equation*}
If the value of $\alpha$ is taken this corresponds to
\begin{equation*}
    F(q_{\alpha-})< \alpha = F(q_{\alpha})
\end{equation*}

\subsubsection{Transforming Random Variables}
\begin{itemize}
    \item Let $X$ be a random variable with cdf $F_X$ and density $f_X$
    \item $Y=g(X)$ for a differentiable strictly increasing transformation $g:\mathbb{R}\to\mathbb{R}$
\end{itemize}
Then
\begin{align*}
    f_Y(y)   & =(g^{-1})'(y)f_X(g^{-1}(y))=\frac{f_X(g^{-1}(y))}{g'(g^{-1}(y))}     \\
    F_{Y}(y) & =\mathbb{P}(g(X)\leq y)=\mathbb{P}(X\leq g^{-1}(y))=F_{X}(g^{-1}(y))
\end{align*}
i.e.\ for $f_Y(y)$ one can choose between 2 formulas depending on whether the derivative of $g$ or $g^{-1}$ is easier to calculate.

\ptitle{Linear Transformation}
\begin{align*}
    g(x)      & =a+bx,\;a\in\mathbb{R}\text,\;b>0           \\
    g^{-1}(y) & =\frac{y-a}b, g'(x)=b                       \\
    f_Y(y)    & =\frac1bf_X\left(\frac{y-a}b\right)         \\
    F_Y(y)    & =F_X(g^{-1}(y))=F_X\left(\frac{y-a}b\right)
\end{align*}

\ptitle{Exponential Transformation}
\begin{align*}
    X      & \sim\mathcal{N}(\mu,\sigma^2)                                                              \\
    Y      & =\exp(X),\;W_Y=(0,\infty) \text{(log-normal distribution)}                                 \\
    g(x)   & =\exp(x),\quad g^{-1}(y)=\log(y),\quad(g^{-1})'(y)=\frac{1}{y}                             \\
    f_Y(y) & =\frac{1}{\sqrt{2\pi\sigma^2}y}\exp\left(-\frac{{(\log(y)-\mu)}^2}{2\sigma^2}\right),\;y>0 \\
    F_Y(y) & =\begin{aligned}\Phi\left(\frac{\log(y)-\mu}\sigma\right),\;y>0\end{aligned}
\end{align*}

\subsubsection{Simulation of Random Variables}

Let $U\sim$ Unif$(0,1)$ and consider a cdf $F$
\begin{equation*}
    X=F^{-1}(U) \text{ has cdf }F
\end{equation*}
because
\noindent\begin{equation*}
    \mathbb{P}(X\leq x)=\mathbb{P}(F^{-1}(U)\leq x)=\mathbb{P}(U\leq F(x))=F(x)
\end{equation*}
as $U$ is a standard uniform distribution on $[0,1]$.

\ptitle{Simulating Discrete Random Variables}

\begin{itemize}
    \item If $F$ is discrete, $F^{-1}$ has to be understood in a generalized sense.
    \item Then $F^{-1}(U)$ still has cdf $F$
    \item As $F(x)$ only takes discrete values, the interval $[F(x_1), F(x_2))$ between two of these discrete values ($F(x_1), F(x_2)$) must be mapped to one and the same $x$. % ChkTex 9
    \item Conversely, one fixed value $F(x_1)$ must be mapped to an interval on the $x$-axis
\end{itemize}

\subsection{Continuous Distributions}
\subsubsection{Uniform Distribution}
Continuous version of the Laplace model

\renewcommand{\arraystretch}{1.3}
\setlength{\oldtabcolsep}{\tabcolsep}\setlength\tabcolsep{0pt}

\begin{tabularx}{\linewidth}{@{}p{0.55\linewidth}p{0.45\linewidth}@{}}
    \begin{minipage}{\linewidth}
        \noindent\begin{flalign*}{
             & X \sim \mathrm{Unif}(a,b),\; a,b\in \mathbb{R},\;a<b                   & \\
             & W=\left[a, b\right]=\left(a, b\right)                                  & \\
             & f(x)=\frac{1}{b-a},x\in[a,b]                                           & \\
             & F(x)=\frac{x-a}{b-a},x\in[a,b]                                         & \\
             & \mathbb{E}[X]=\frac{a+b}{2}, \; \mathrm{Var}(X)=\frac{{(b-a)}^2}{12} &
            }\end{flalign*}
    \end{minipage}
     &
    \begin{minipage}{\linewidth}
        \begin{tikzpicture}
    \tiny
    \begin{axis}[
            width = 5cm,
            height = 3cm,
            % width = \linewidth,
            % unit vector ratio={1 1},
            name = axis1,
            xmin = -2,
            ymin  = 0,
            xmax = 3,
            ymax = 1,
            xtick distance=1,
            % ytick={0,3,8},
            % yticklabels={0,$\lambda_1$,$\lambda_2$},
            % xtick = {0},
            % xticklabel=\empty,
            xlabel={$x$},
            ylabel={PDF $f(x)$},
            legend style={at={(1,1)},anchor=north east},
            legend style={font=\tiny},
            grid style=dashed,
        ]
        \addplot [
            color=red,
            line width = 1pt]
        coordinates {
                (-2,0)
                (-1,0)
            };
        \addplot [
            color=red,
            line width = 1pt]
        coordinates {
                (-1,0)
                (-1,0.33)
            };
        \addplot [
            color=red,
            line width = 1pt]
        coordinates {
                (-1,0.33)
                (2, 0.33)
            };
        \addplot [
            color=red,
            line width = 1pt]
        coordinates {
                (2, 0.33)
                (2, 0)
            };
        \addplot [
            color=red,
            line width = 1pt]
        coordinates {
                (2, 0)
                (3, 0)
            };
        \addlegendentry{$a = -1; b = 2$}
    \end{axis}
    \begin{axis}[
            width = 5cm,
            height = 3cm,
            % width = \linewidth,
            % unit vector ratio={1 1},
            name = axis2,
            xmin = -2,
            ymin  = 0,
            xmax = 3,
            ymax = 1,
            xtick distance=1,
            % ytick={0,3,8},
            % yticklabels={0,$\lambda_1$,$\lambda_2$},
            % xtick = {0},
            % xticklabel=\empty,
            at=(axis1.below south west), anchor=above north west,
            xlabel={$x$},
            ylabel={CDF $F(x)$},
            legend style={at={(1,0)},anchor=south east},
            legend style={font=\tiny},
            grid style=dashed,
        ]
        \addplot [
            color=red,
            line width = 1pt]
        coordinates {
                (-2,0)
                (-1,0)
            };
        \addplot [
            color=red,
            line width = 1pt]
        coordinates {
                (-1,0)
                (2,1)
            };
        \addplot [
            color=red,
            line width = 1pt]
        coordinates {
                (2, 1)
                (3, 1)
            };
    \end{axis}
\end{tikzpicture}
    \end{minipage}
\end{tabularx}

\renewcommand{\arraystretch}{1}
\setlength\tabcolsep{\oldtabcolsep}

\subsubsection{Beta Distribution}

\renewcommand{\arraystretch}{1.3}
\setlength{\oldtabcolsep}{\tabcolsep}\setlength\tabcolsep{0pt}

\begin{tabularx}{\linewidth}{@{}p{0.57\linewidth}p{0.43\linewidth}@{}}
    \begin{minipage}[t]{\linewidth}
        \noindent\begin{flalign*}{
             & X \sim \mathrm{Beta}(\alpha, \beta), \;\alpha, \beta>0                                     & \\
             & W=[0,1]                                                                                    & \\
             & f(x)=\frac{\Gamma(\alpha)\Gamma(\beta)}{\Gamma(\alpha+\beta)}x^{\alpha-1}{(1-x)}^{\beta-1} & \\
             & F(x)=\int_0^x f(u)du,\;x\geq0                                                              & \\
             & \mathbb{E}[X]=\int_0^1 xf(x)dx=\frac{\alpha}{\alpha + \beta}                               & \\
             & \mathrm{Var}(X)=\frac{\alpha\beta}{{(\alpha+\beta)}^2(\alpha+\beta+1)}
            }\end{flalign*}
    \end{minipage}
     &
    \includegraphics[width=0.9\linewidth, align=t]{Cont_Beta_Distribution.png}
    \\
\end{tabularx}

\renewcommand{\arraystretch}{1}
\setlength\tabcolsep{\oldtabcolsep}

\begin{itemize}
    \item Can be rescaled to arbitrary intervals which then of course affects the formulas for $\mathbb{E}[X]$ and others.
    \item$\mathrm{Beta(1,1)} = \mathrm{Unif}(0,1)$
\end{itemize}

\subsubsection{Standard Normal Distribution}

\renewcommand{\arraystretch}{1.3}
\setlength{\oldtabcolsep}{\tabcolsep}\setlength\tabcolsep{0pt}

\begin{tabularx}{\linewidth}{@{}p{0.55\linewidth}p{0.45\linewidth}@{}}
    \begin{minipage}{\linewidth}
        \noindent\begin{flalign*}{
             & Z\sim\mathcal{N}(0,1),\;\mu = 0,\; \sigma^2 = 1           & \\
             & W=\mathbb{R}                                              & \\
             & \varphi(x)=f(x)=\frac{1}{\sqrt{2\pi}}e^{-\frac{x^2}{2}}   & \\
             & \Phi(z)=\mathbb{P}(Z\leq z)= \int_{-\infty}^z\varphi(x)dx & \\
             & \mathbb{E}[Z] = 0, \qquad \mathrm{Var}(Z)=1               &
            }\end{flalign*}
    \end{minipage}
     &
    \begin{minipage}{\linewidth}
        \begin{tikzpicture}
    \tiny
    \begin{axis}[
            width = 5cm,
            height = 3cm,
            % width = \linewidth,
            % unit vector ratio={1 1},
            name = axis1,
            xmin = -4,
            ymin  = 0,
            xmax = 4,
            ymax = 0.5,
            xtick distance=1,
            % ytick={0,3,8},
            % yticklabels={0,$\lambda_1$,$\lambda_2$},
            % xtick = {0},
            % xticklabel=\empty,
            xlabel={$x$},
            ylabel={PDF $f(x)$},
            legend style={at={(1,1)},anchor=north east},
            legend style={font=\tiny},
            grid style=dashed,
            domain=-4:4,
            samples=50,
        ]
        \addplot [
            color=red,
            line width = 1pt,
        ]
        {1/(sqrt(2*pi))*exp(-x^2/2)};
    \end{axis}
    \begin{axis}[
            width = 5cm,
            height = 3cm,
            % width = \linewidth,
            % unit vector ratio={1 1},
            name = axis2,
            xmin = -4,
            ymin  = 0,
            xmax = 4,
            ymax = 1,
            xtick distance=1,
            % ytick={0,3,8},
            % yticklabels={0,$\lambda_1$,$\lambda_2$},
            % xtick = {0},
            % xticklabel=\empty,
            at=(axis1.below south west), anchor=above north west,
            xlabel={$x$},
            ylabel={CDF $F(x)$},
            legend style={at={(1,0)},anchor=south east},
            legend style={font=\tiny},
            grid style=dashed,
        ]
        \addplot [
            color=red,
            line width = 1pt]
        coordinates {
                (-4, 0.00003)
                (-3.9, 0.00005)
                (-3.8, 0.00007)
                (-3.7, 0.00011)
                (-3.6, 0.00016)
                (-3.5, 0.00023)
                (-3.4, 0.00034)
                (-3.3, 0.00048)
                (-3.2, 0.00069)
                (-3.1, 0.00097)
                (-3.0, 0.00135)
                (-2.9, 0.00187)
                (-2.8, 0.00256)
                (-2.7, 0.00347)
                (-2.6, 0.00466)
                (-2.5, 0.00621)
                (-2.4, 0.0082)
                (-2.3, 0.01072)
                (-2.2, 0.0139)
                (-2.1, 0.01786)
                (-2.0, 0.02275)
                (-1.9, 0.02872)
                (-1.8, 0.03593)
                (-1.7, 0.04457)
                (-1.6, 0.0548)
                (-1.5, 0.06681)
                (-1.4, 0.08076)
                (-1.3, 0.0968)
                (-1.2, 0.11507)
                (-1.1, 0.13567)
                (-1.0, 0.15866)
                (-0.9, 0.18406)
                (-0.8, 0.21186)
                (-0.7, 0.24196)
                (-0.6, 0.27425)
                (-0.5, 0.30854)
                (-0.4, 0.34458)
                (-0.3, 0.38209)
                (-0.2, 0.42074)
                (-0.1, 0.46017)
                (0.0, 0.5)
                (0.1, 0.53983)
                (0.2, 0.57926)
                (0.3, 0.61791)
                (0.4, 0.65542)
                (0.5, 0.69146)
                (0.6, 0.72575)
                (0.7, 0.75804)
                (0.8, 0.78814)
                (0.9, 0.81594)
                (1.0, 0.84134)
                (1.1, 0.86433)
                (1.2, 0.88493)
                (1.3, 0.9032)
                (1.4, 0.91924)
                (1.5, 0.93319)
                (1.6, 0.9452)
                (1.7, 0.95543)
                (1.8, 0.96407)
                (1.9, 0.97128)
                (2.0, 0.97725)
                (2.1, 0.98214)
                (2.2, 0.9861)
                (2.3, 0.98928)
                (2.4, 0.9918)
                (2.5, 0.99379)
                (2.6, 0.99534)
                (2.7, 0.99653)
                (2.8, 0.99744)
                (2.9, 0.99813)
                (3.0, 0.99865)
                (3.1, 0.99903)
                (3.2, 0.99931)
                (3.3, 0.99952)
                (3.4, 0.99966)
                (3.5, 0.99977)
                (3.6, 0.99984)
                (3.7, 0.99989)
                (3.8, 0.99993)
                (3.9, 0.99995)
                (4.0, 0.99997)
            };
    \end{axis}
\end{tikzpicture}
    \end{minipage}
\end{tabularx}

\renewcommand{\arraystretch}{1}
\setlength\tabcolsep{\oldtabcolsep}

\subsubsection{Normal Distribution}

\renewcommand{\arraystretch}{1.3}
\setlength{\oldtabcolsep}{\tabcolsep}\setlength\tabcolsep{0pt}

\begin{tabularx}{\linewidth}{@{}p{0.55\linewidth}p{0.45\linewidth}@{}}
    \begin{minipage}{\linewidth}
        % $Z$ is a random variable of the standard normal distribution.
        \noindent\begin{flalign*}{
             & X=\mu+\sigma Z\sim\mathcal{N}(\mu,\sigma^2)                                               & \\
             & \mu\in \mathbb{R},\;\sigma^2  > 0                                                         & \\
             & W=\mathbb{R}                                                                              & \\
             & f(x) = \frac{1}{\sqrt{2\pi\sigma^2}}\exp\left(-\frac{{(x-\mu)}^2}{2\sigma^2}\right)       & \\
             & F(x)=\mathbb{P}(\mu+\sigma Z\leq x)                                                       & \\
             & \;=\mathbb{P}\left(Z\leq\frac{x-\mu}{\sigma}\right)=\Phi\left(\frac{x-\mu}{\sigma}\right) & \\
             & \mathbb{E}[X] = \mu, \qquad \mathrm{Var}(X)=\sigma^2                                      &
            }\end{flalign*}
    \end{minipage}
     &
    \begin{minipage}{\linewidth}
        \begin{tikzpicture}
    \tiny
    \begin{axis}[
            width = 6cm,
            height = 3.6cm,
            % width = \linewidth,
            % unit vector ratio={1 1},
            name = axis1,
            xmin = -9,
            ymin  = 0,
            xmax = 5,
            ymax = 0.7,
            xtick distance=1,
            % ytick={0,3,8},
            % yticklabels={0,$\lambda_1$,$\lambda_2$},
            % xtick = {0},
            % xticklabel=\empty,
            xlabel={$x$},
            ylabel={PDF $f(x)$},
            legend style={at={(1,1)},anchor=north east},
            legend style={font=\tiny},
            grid style=dashed,
            domain=-9:5,
            samples=50,
        ]
        \addplot [
            color=red,
            line width = 1pt,
        ]
        {1/(sqrt(2*pi))*exp(-x^2/2)};
        \addlegendentry{$\mu = 0; \sigma = 1$}
        \addplot [
            color=blue,
            line width = 1pt,
        ]
        {1/(sqrt(2*pi*(-2)^2))*exp(-(x+2)^2/(2*(-2)^2))};
        \addlegendentry{$\mu = -2; \sigma = 2$}
    \end{axis}
    \begin{axis}[
            width = 6cm,
            height = 3.6cm,
            % width = \linewidth,
            % unit vector ratio={1 1},
            name = axis2,
            xmin = -9,
            ymin  = 0,
            xmax = 5,
            ymax = 1,
            xtick distance=1,
            % ytick={0,3,8},
            % yticklabels={0,$\lambda_1$,$\lambda_2$},
            % xtick = {0},
            % xticklabel=\empty,
            at=(axis1.below south west), anchor=above north west,
            xlabel={$x$},
            ylabel={CDF $F(x)$},
            legend style={at={(1,0)},anchor=south east},
            legend style={font=\tiny},
            grid style=dashed,
        ]
        \addplot [
            color=red,
            line width = 1pt]
        coordinates {
                (-4.0, 0.00003)
                (-3.9, 0.00005)
                (-3.8, 0.00007)
                (-3.7, 0.00011)
                (-3.6, 0.00016)
                (-3.5, 0.00023)
                (-3.4, 0.00034)
                (-3.3, 0.00048)
                (-3.2, 0.00069)
                (-3.1, 0.00097)
                (-3.0, 0.00135)
                (-2.9, 0.00187)
                (-2.8, 0.00256)
                (-2.7, 0.00347)
                (-2.6, 0.00466)
                (-2.5, 0.00621)
                (-2.4, 0.0082)
                (-2.3, 0.01072)
                (-2.2, 0.0139)
                (-2.1, 0.01786)
                (-2.0, 0.02275)
                (-1.9, 0.02872)
                (-1.8, 0.03593)
                (-1.7, 0.04457)
                (-1.6, 0.0548)
                (-1.5, 0.06681)
                (-1.4, 0.08076)
                (-1.3, 0.0968)
                (-1.2, 0.11507)
                (-1.1, 0.13567)
                (-1.0, 0.15866)
                (-0.9, 0.18406)
                (-0.8, 0.21186)
                (-0.7, 0.24196)
                (-0.6, 0.27425)
                (-0.5, 0.30854)
                (-0.4, 0.34458)
                (-0.3, 0.38209)
                (-0.2, 0.42074)
                (-0.1, 0.46017)
                (0.0, 0.5)
                (0.1, 0.53983)
                (0.2, 0.57926)
                (0.3, 0.61791)
                (0.4, 0.65542)
                (0.5, 0.69146)
                (0.6, 0.72575)
                (0.7, 0.75804)
                (0.8, 0.78814)
                (0.9, 0.81594)
                (1.0, 0.84134)
                (1.1, 0.86433)
                (1.2, 0.88493)
                (1.3, 0.9032)
                (1.4, 0.91924)
                (1.5, 0.93319)
                (1.6, 0.9452)
                (1.7, 0.95543)
                (1.8, 0.96407)
                (1.9, 0.97128)
                (2.0, 0.97725)
                (2.1, 0.98214)
                (2.2, 0.9861)
                (2.3, 0.98928)
                (2.4, 0.9918)
                (2.5, 0.99379)
                (2.6, 0.99534)
                (2.7, 0.99653)
                (2.8, 0.99744)
                (2.9, 0.99813)
                (3.0, 0.99865)
                (3.1, 0.99903)
                (3.2, 0.99931)
                (3.3, 0.99952)
                (3.4, 0.99966)
                (3.5, 0.99977)
                (3.6, 0.99984)
                (3.7, 0.99989)
                (3.8, 0.99993)
                (3.9, 0.99995)
                (4.0, 0.99997)
            };
        \addplot [
            color=blue,
            line width = 1pt]
        coordinates {
                (-4.0*2-2, 0.00003)
                (-3.9*2-2, 0.00005)
                (-3.8*2-2, 0.00007)
                (-3.7*2-2, 0.00011)
                (-3.6*2-2, 0.00016)
                (-3.5*2-2, 0.00023)
                (-3.4*2-2, 0.00034)
                (-3.3*2-2, 0.00048)
                (-3.2*2-2, 0.00069)
                (-3.1*2-2, 0.00097)
                (-3.0*2-2, 0.00135)
                (-2.9*2-2, 0.00187)
                (-2.8*2-2, 0.00256)
                (-2.7*2-2, 0.00347)
                (-2.6*2-2, 0.00466)
                (-2.5*2-2, 0.00621)
                (-2.4*2-2, 0.0082)
                (-2.3*2-2, 0.01072)
                (-2.2*2-2, 0.0139)
                (-2.1*2-2, 0.01786)
                (-2.0*2-2, 0.02275)
                (-1.9*2-2, 0.02872)
                (-1.8*2-2, 0.03593)
                (-1.7*2-2, 0.04457)
                (-1.6*2-2, 0.0548)
                (-1.5*2-2, 0.06681)
                (-1.4*2-2, 0.08076)
                (-1.3*2-2, 0.0968)
                (-1.2*2-2, 0.11507)
                (-1.1*2-2, 0.13567)
                (-1.0*2-2, 0.15866)
                (-0.9*2-2, 0.18406)
                (-0.8*2-2, 0.21186)
                (-0.7*2-2, 0.24196)
                (-0.6*2-2, 0.27425)
                (-0.5*2-2, 0.30854)
                (-0.4*2-2, 0.34458)
                (-0.3*2-2, 0.38209)
                (-0.2*2-2, 0.42074)
                (-0.1*2-2, 0.46017)
                (0.0*2-2, 0.5)
                (0.1*2-2, 0.53983)
                (0.2*2-2, 0.57926)
                (0.3*2-2, 0.61791)
                (0.4*2-2, 0.65542)
                (0.5*2-2, 0.69146)
                (0.6*2-2, 0.72575)
                (0.7*2-2, 0.75804)
                (0.8*2-2, 0.78814)
                (0.9*2-2, 0.81594)
                (1.0*2-2, 0.84134)
                (1.1*2-2, 0.86433)
                (1.2*2-2, 0.88493)
                (1.3*2-2, 0.9032)
                (1.4*2-2, 0.91924)
                (1.5*2-2, 0.93319)
                (1.6*2-2, 0.9452)
                (1.7*2-2, 0.95543)
                (1.8*2-2, 0.96407)
                (1.9*2-2, 0.97128)
                (2.0*2-2, 0.97725)
                (2.1*2-2, 0.98214)
                (2.2*2-2, 0.9861)
                (2.3*2-2, 0.98928)
                (2.4*2-2, 0.9918)
                (2.5*2-2, 0.99379)
                (2.6*2-2, 0.99534)
                (2.7*2-2, 0.99653)
                (2.8*2-2, 0.99744)
                (2.9*2-2, 0.99813)
                (3.0*2-2, 0.99865)
                (3.1*2-2, 0.99903)
                (3.2*2-2, 0.99931)
                (3.3*2-2, 0.99952)
                (3.4*2-2, 0.99966)
                (3.5*2-2, 0.99977)
                (3.6*2-2, 0.99984)
                (3.7*2-2, 0.99989)
                (3.8*2-2, 0.99993)
                (3.9*2-2, 0.99995)
                (4.0*2-2, 0.99997)
            };
    \end{axis}
\end{tikzpicture}
    \end{minipage} \\
\end{tabularx}

\renewcommand{\arraystretch}{1}
\setlength\tabcolsep{\oldtabcolsep}


\subsubsection{Exponential Distribution}

\renewcommand{\arraystretch}{1.3}
\setlength{\oldtabcolsep}{\tabcolsep}\setlength\tabcolsep{0pt}

\begin{tabularx}{\linewidth}{@{}p{0.45\linewidth}p{0.55\linewidth}@{}}
    \begin{minipage}[t]{\linewidth}
        \noindent\begin{flalign*}{
             & X \sim \mathrm{Exp}(\lambda), \;\lambda>0                                    & \\
             & W=[0,\infty)                                                                 & \\ % ChkTex 9
             & f(x)=\mathbb{I}_{\{x\geq0\}}\lambda e^{-\lambda x}                           & \\
             & \mathbb{I}_{\{x\geq0\}} \text{: unit step function}                          & \\
             & F(x)=\mathbb{I}-e^{-\lambda x},\;x\geq0                                      & \\
             & \mathbb{E}[X] = \frac{1}{\lambda}, \quad \mathrm{Var}(X)=\frac{1}{\lambda^2} &
            }\end{flalign*}
    \end{minipage}
     &
    \includegraphics[width=0.8\linewidth, align=t]{Cont_Exponential_Distribution.png}
\end{tabularx}

\renewcommand{\arraystretch}{1}
\setlength\tabcolsep{\oldtabcolsep}

\begin{itemize}
    \item Simplest model for waiting or life times
    \item \textbf{memoryless} on $\mathbb{R}^+$ i.e.\ waiting for $\Delta t$ again is equally likely after having already waited for $\Delta t$.
    \item If waiting times are $\mathrm{Exp}(\lambda)$, then there are $\mathrm{Pois}(\lambda t)$ events in intervals of length $t$
\end{itemize}


\subsubsection{Gamma Distribution}

\renewcommand{\arraystretch}{1.3}
\setlength{\oldtabcolsep}{\tabcolsep}\setlength\tabcolsep{0pt}

\begin{tabularx}{\linewidth}{@{}p{0.5\linewidth}p{0.49\linewidth}@{}}
    \begin{minipage}[t]{\linewidth}
        \noindent\begin{flalign*}{
             & X \sim \mathrm{Gamma}(\alpha, \beta), \;\alpha, \beta>0                                                             & \\
             & W=[0,\infty)                                                                                                        & \\ % ChkTex 9
             & f(x)=\mathbb{I}_{\{x>0\}}\underbrace{\frac{\beta^\alpha}{\Gamma(\alpha)}}_{\mathsf{norm.}} x^{\alpha-1}e^{-\beta x} & \\
             & \Gamma(\alpha)=\int_0^\infty x^{\alpha-1}e^{-x}dx                                                                   & \\
             & F(x)=\int_0^x f(u)du,\;x\geq0                                                                                       & \\
             & \mathbb{E}[X]=\frac{\alpha}{\beta}                                                                                  & \\
             & \mathrm{Var}(X)=\frac{\alpha}{\beta^2}                                                                              &
            }\end{flalign*}
    \end{minipage}
     &
    \includegraphics[width=0.99\linewidth, align=t]{Cont_Gamma_Distribution.png} \\
\end{tabularx}

\renewcommand{\arraystretch}{1}
\setlength\tabcolsep{\oldtabcolsep}

\begin{itemize}
    \item $\mathrm{Gamma}(1, \beta)=\mathrm{Exp}(\beta)$
    \item $\mathrm{Gamma}(\alpha, \beta)$ is a more general model for waiting times than Exp$( \lambda)$.
    \item It is also used to model severity of insurance claims.
\end{itemize}

